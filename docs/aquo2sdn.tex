% Options for packages loaded elsewhere
\PassOptionsToPackage{unicode}{hyperref}
\PassOptionsToPackage{hyphens}{url}
%
\documentclass[
]{book}
\usepackage{amsmath,amssymb}
\usepackage{lmodern}
\usepackage{iftex}
\ifPDFTeX
  \usepackage[T1]{fontenc}
  \usepackage[utf8]{inputenc}
  \usepackage{textcomp} % provide euro and other symbols
\else % if luatex or xetex
  \usepackage{unicode-math}
  \defaultfontfeatures{Scale=MatchLowercase}
  \defaultfontfeatures[\rmfamily]{Ligatures=TeX,Scale=1}
\fi
% Use upquote if available, for straight quotes in verbatim environments
\IfFileExists{upquote.sty}{\usepackage{upquote}}{}
\IfFileExists{microtype.sty}{% use microtype if available
  \usepackage[]{microtype}
  \UseMicrotypeSet[protrusion]{basicmath} % disable protrusion for tt fonts
}{}
\makeatletter
\@ifundefined{KOMAClassName}{% if non-KOMA class
  \IfFileExists{parskip.sty}{%
    \usepackage{parskip}
  }{% else
    \setlength{\parindent}{0pt}
    \setlength{\parskip}{6pt plus 2pt minus 1pt}}
}{% if KOMA class
  \KOMAoptions{parskip=half}}
\makeatother
\usepackage{xcolor}
\IfFileExists{xurl.sty}{\usepackage{xurl}}{} % add URL line breaks if available
\IfFileExists{bookmark.sty}{\usepackage{bookmark}}{\usepackage{hyperref}}
\hypersetup{
  pdftitle={AQUO2SDN mapping},
  pdfauthor={Willem Stolte},
  hidelinks,
  pdfcreator={LaTeX via pandoc}}
\urlstyle{same} % disable monospaced font for URLs
\usepackage{color}
\usepackage{fancyvrb}
\newcommand{\VerbBar}{|}
\newcommand{\VERB}{\Verb[commandchars=\\\{\}]}
\DefineVerbatimEnvironment{Highlighting}{Verbatim}{commandchars=\\\{\}}
% Add ',fontsize=\small' for more characters per line
\usepackage{framed}
\definecolor{shadecolor}{RGB}{248,248,248}
\newenvironment{Shaded}{\begin{snugshade}}{\end{snugshade}}
\newcommand{\AlertTok}[1]{\textcolor[rgb]{0.94,0.16,0.16}{#1}}
\newcommand{\AnnotationTok}[1]{\textcolor[rgb]{0.56,0.35,0.01}{\textbf{\textit{#1}}}}
\newcommand{\AttributeTok}[1]{\textcolor[rgb]{0.77,0.63,0.00}{#1}}
\newcommand{\BaseNTok}[1]{\textcolor[rgb]{0.00,0.00,0.81}{#1}}
\newcommand{\BuiltInTok}[1]{#1}
\newcommand{\CharTok}[1]{\textcolor[rgb]{0.31,0.60,0.02}{#1}}
\newcommand{\CommentTok}[1]{\textcolor[rgb]{0.56,0.35,0.01}{\textit{#1}}}
\newcommand{\CommentVarTok}[1]{\textcolor[rgb]{0.56,0.35,0.01}{\textbf{\textit{#1}}}}
\newcommand{\ConstantTok}[1]{\textcolor[rgb]{0.00,0.00,0.00}{#1}}
\newcommand{\ControlFlowTok}[1]{\textcolor[rgb]{0.13,0.29,0.53}{\textbf{#1}}}
\newcommand{\DataTypeTok}[1]{\textcolor[rgb]{0.13,0.29,0.53}{#1}}
\newcommand{\DecValTok}[1]{\textcolor[rgb]{0.00,0.00,0.81}{#1}}
\newcommand{\DocumentationTok}[1]{\textcolor[rgb]{0.56,0.35,0.01}{\textbf{\textit{#1}}}}
\newcommand{\ErrorTok}[1]{\textcolor[rgb]{0.64,0.00,0.00}{\textbf{#1}}}
\newcommand{\ExtensionTok}[1]{#1}
\newcommand{\FloatTok}[1]{\textcolor[rgb]{0.00,0.00,0.81}{#1}}
\newcommand{\FunctionTok}[1]{\textcolor[rgb]{0.00,0.00,0.00}{#1}}
\newcommand{\ImportTok}[1]{#1}
\newcommand{\InformationTok}[1]{\textcolor[rgb]{0.56,0.35,0.01}{\textbf{\textit{#1}}}}
\newcommand{\KeywordTok}[1]{\textcolor[rgb]{0.13,0.29,0.53}{\textbf{#1}}}
\newcommand{\NormalTok}[1]{#1}
\newcommand{\OperatorTok}[1]{\textcolor[rgb]{0.81,0.36,0.00}{\textbf{#1}}}
\newcommand{\OtherTok}[1]{\textcolor[rgb]{0.56,0.35,0.01}{#1}}
\newcommand{\PreprocessorTok}[1]{\textcolor[rgb]{0.56,0.35,0.01}{\textit{#1}}}
\newcommand{\RegionMarkerTok}[1]{#1}
\newcommand{\SpecialCharTok}[1]{\textcolor[rgb]{0.00,0.00,0.00}{#1}}
\newcommand{\SpecialStringTok}[1]{\textcolor[rgb]{0.31,0.60,0.02}{#1}}
\newcommand{\StringTok}[1]{\textcolor[rgb]{0.31,0.60,0.02}{#1}}
\newcommand{\VariableTok}[1]{\textcolor[rgb]{0.00,0.00,0.00}{#1}}
\newcommand{\VerbatimStringTok}[1]{\textcolor[rgb]{0.31,0.60,0.02}{#1}}
\newcommand{\WarningTok}[1]{\textcolor[rgb]{0.56,0.35,0.01}{\textbf{\textit{#1}}}}
\usepackage{longtable,booktabs,array}
\usepackage{calc} % for calculating minipage widths
% Correct order of tables after \paragraph or \subparagraph
\usepackage{etoolbox}
\makeatletter
\patchcmd\longtable{\par}{\if@noskipsec\mbox{}\fi\par}{}{}
\makeatother
% Allow footnotes in longtable head/foot
\IfFileExists{footnotehyper.sty}{\usepackage{footnotehyper}}{\usepackage{footnote}}
\makesavenoteenv{longtable}
\usepackage{graphicx}
\makeatletter
\def\maxwidth{\ifdim\Gin@nat@width>\linewidth\linewidth\else\Gin@nat@width\fi}
\def\maxheight{\ifdim\Gin@nat@height>\textheight\textheight\else\Gin@nat@height\fi}
\makeatother
% Scale images if necessary, so that they will not overflow the page
% margins by default, and it is still possible to overwrite the defaults
% using explicit options in \includegraphics[width, height, ...]{}
\setkeys{Gin}{width=\maxwidth,height=\maxheight,keepaspectratio}
% Set default figure placement to htbp
\makeatletter
\def\fps@figure{htbp}
\makeatother
\setlength{\emergencystretch}{3em} % prevent overfull lines
\providecommand{\tightlist}{%
  \setlength{\itemsep}{0pt}\setlength{\parskip}{0pt}}
\setcounter{secnumdepth}{5}
\usepackage{booktabs}
\usepackage{tabu}
\ifLuaTeX
  \usepackage{selnolig}  % disable illegal ligatures
\fi
\usepackage[]{natbib}
\bibliographystyle{apalike}

\title{AQUO2SDN mapping}
\author{Willem Stolte}
\date{2022-07-19}

\begin{document}
\maketitle

{
\setcounter{tocdepth}{1}
\tableofcontents
}
\hypertarget{voorwoord}{%
\chapter{Voorwoord}\label{voorwoord}}

Uitwisseling van watergerelateerde data met andere landen is belangrijk voor analyses en beoordelingen van internationale wateren. Van oudsher is er veel uitwisseling van mariene data. Sinds SeaDataNet (\href{https://www.seadatanet.org/}{SDN}) zijn hiervoor internationale standaarden gebruikt en ontwikkeld. Als uitgangspunt is het systeem van de Brittisch Oceanographic Data Center (\href{https://www.bodc.ac.uk/}{BODC}) genomen. Dit heeft geleid tot een set internationaal geaccepteerde semantische standaarden (vocabulaires) voor registratie van in situ metingen in het mariene domein.

Er zijn binnen SeaDataNet ook technische standaarden ontwikkeld voor het zoeken, vinden en transporteren van metadata en data. Het gangbare formaat voor metadata is de Common Data Index (CDI), en voor data het Ocean Data View (ODV) formaat.

De nederlandse standaard voor waterdata is \href{https://www.aquo.nl/index.php/Hoofdpagina}{AQUO}. Er is op dit moment geen eenduidigen en complete mapping om AQUO-gestandaardiseerde data te transformeren naar SDN-gestandaardiseerde data. De technische standaard voor uitlevering van data variëert.

We beperken ons in dit project tot het mappen van de semantische standaarden.

\hypertarget{aquo}{%
\chapter{AQUO}\label{aquo}}

AQUO is de nederlandse standaard voor watergerelateerde data. Voor uitwisseling van meetgegevens is er het informatiemodel IM-AQUO. Dit model beschrijft voor in situ metingen de ``wat'' en ``hoe''.

\hypertarget{relevante-aquo-vocabulaires-domeintabellen}{%
\section{Relevante AQUO vocabulaires (domeintabellen)}\label{relevante-aquo-vocabulaires-domeintabellen}}

De belangrijkste tabellen voor een goede mapping van in situ waarnemingen zijn:

\begin{description}
\item[Grootheid of Typering]
Beschrijving en code van de grootheid (quantitatief) of typering (kwalitatief). Voor elke grootheid is een of meerdere Eenheden beschikbaar in de Eenheid tabel.
\item[Parameter]
Dit is een samengestelde tabel/veld. In AQUO is dit onderverdeeld in Chemische Stof, Taxonnaam, of
\item[Hoedanigheid]
Deze tabel bevat diverse informatie die te maken heeft met bijvoorbeeld monstervoorberking (opgeloste of deeltjesgebonden fractie), grootteklassen, leeftijdsklasse.
\item[Monstercriterium]
Deze tabel bevat informatie over overige criteria, bijvoorbeeld in een fractie kleiner dan een bepaalde korrelgrootte. In de datadistributielaag zijn de velden Hoedanigheid en Monstercriterium samengevoegd in een veld (genaamd Hoedanigheid)
\item[Eenheid]
Gebruikte eenheid. Elke eenheid is gekoppeld aan een of meerdere grootheden.
\end{description}

\begin{verbatim}
#> # A tibble: 0 x 7
#> # ... with 7 variables: domeintabel <chr>,
#> #   domeintabelsoort <chr>, wijzigingsdatum <date>,
#> #   begin_geldigheid <date>, eind_geldigheid <date>,
#> #   kolommen <list>, guid <chr>
#> # A tibble: 0 x 10
#> # ... with 10 variables: id <dbl>, codes <chr>,
#> #   omschrijving <chr>, begin_geldigheid <date>,
#> #   eind_geldigheid <date>, guid <chr>, dimensie <chr>,
#> #   gerelateerd <chr>, groep <chr>, omrekenfactor <chr>
\end{verbatim}

\hypertarget{gebruik-van-aquo-in-rws-datadistributielaag}{%
\section{Gebruik van AQUO in RWS datadistributielaag}\label{gebruik-van-aquo-in-rws-datadistributielaag}}

\hypertarget{bodc-semantic-standard}{%
\chapter{BODC semantic standard}\label{bodc-semantic-standard}}

De Brittisch Oceanographic Data Centre (\href{https://www.bodc.ac.uk/}{BODC}) is in 1969 ingesteld door de Natural Environmental Research Council (\href{https://www.ukri.org/councils/nerc/}{NERC}), en beheert een set vocabulaires voor gebruik in het mariene domein. Bij de oprichting van \href{https://www.seadatanet.org/}{SeaDataNet}, een infrastructuur voor het zoeken, vinden en distribueren van in situ data, zijn deze vocabulaires geadopteerd als ``\href{https://www.seadatanet.org/Standards/Common-Vocabularies}{common vocabularies}''voor gebruik als semantische standaard. SeaDataNet houdt daarnaast catalogi bij van \href{https://www.seadatanet.org/Standards/Metadata-formats}{metadataformats} voor bijv cruise rapporten (\href{https://www.seadatanet.org/Standards/Metadata-formats/CSR}{CSR}), projecten (\href{https://www.seadatanet.org/Standards/Metadata-formats/EDMERP}{EDMERP}), datasets (\href{https://www.seadatanet.org/Standards/Metadata-formats/EDMED}{EDMED}) en platforms (\href{https://www.seadatanet.org/Standards/Metadata-formats/EDIOS}{EDIOS}).

Voor de beschrijving van parameter namen is indertijd de P01 vocabulaire opgezet. Dit is een verzameling van beschrijvinde namen die de grootheid, parameter, eventuele details in de bemonstering en/of meetmethode en de matrix vastlegt. De tabel is ``organisch'' gegroeid doordat veel verschillende Europese databeheerders hieraan hun eigen specifieke waarde aan toevoegden. Later is de P01 tabel onderverdeeld in de verschillende onderdelen (``S-tabellen''). De relatie hiertussen is vastgelegd in de ``one-armed bandit'' tabel. Merk op dat door het combineren van de verschillende S-tabelwaarden in principe nieuwe P01 namen gemaakt kunnen worden. Een mapping op het niveau van de ``S''-componenten kan daarom leiden tot niet-bestaande P01 namen.

\hypertarget{de-wat-beschrijving---koppeling-p01-met-s-tabellen-one-armed-bandit}{%
\section{De ``wat'' beschrijving - koppeling P01 met S tabellen (one-armed bandit)}\label{de-wat-beschrijving---koppeling-p01-met-s-tabellen-one-armed-bandit}}

De relatie tussen de primaire parameternamen (P01) en de afzonderlijke componenten voor bestaande P01 namen kan interactief doorzocht worden in \href{https://vocab.seadatanet.org/p01-facet-search?\&step_more=6}{deze link}.

\begin{table}

\caption{\label{tab:p01s-tabel}Voorbeeld 10 elementen uit de tabel [P01 VOCABULARY - FACET SEARCH ON SEMANTIC COMPONENTS]. }
\centering
\begin{tabular}[t]{r|l|l|l|l|l|l|l|l|l|l|l|l|l|l|l|l|l|l|l|l|l|l|l|l|l|l}
\hline
n\_code & P01\_conceptid & P01\_preflabel & S06\_conceptid & S06\_preflabel & S07\_conceptid & S07\_preflabel & s27\_conceptid & S27\_preflabel & S27\_altlabel & CAS no & S02\_conceptid & S02\_preflabel & S26\_conceptid & S26\_preflabel & S25\_conceptid & S25\_preflabel & S03\_conceptid & S03\_preflabel & S04\_conceptid & S04\_preflabel & S05\_conceptid & S05\_preflabel & P02\_conceptid & P02\_preflabel & S21\_conceptid & S21\_preflabel\\
\hline
30895 & PRSCPSTW & Proportion of particles (5-15um) in the sediment by particle sizer &  &  &  &  &  &  &  &  & S028 & in the & MAT00136 & sediment &  &  &  &  &  &  &  &  & MNGS & Sediment grain size parameters & NA & NA\\
\hline
16597 & IC000065 & Concentration of 2,2',4,5,5'-pentachlorobiphenyl \{PCB101 CAS 37680-73-2\} per unit dry weight of sediment <1000um & S0600045 & Concentration &  &  & CS001803 & 2,2',4,5,5'-pentachlorobiphenyl & PCB101 & 37680-73-2 & S041 & per unit dry weight of & MAT01975 & sediment <1000um &  &  &  &  &  &  &  &  & SPCB & Concentration of polychlorobiphenyls (PCBs) in sediment samples & NA & NA\\
\hline
38853 & WB000192 & Concentration of lipids per unit wet weight of biota \{Tursiops truncatus (ITIS: 180426: WoRMS 137111) [Sex: female Subcomponent: liver]\} & S0600045 & Concentration &  &  & CS002806 & lipids &  &  & S055 & per unit wet weight of & MAT01963 & biota & BE007873 & Tursiops truncatus (ITIS: 180426: WoRMS 137111) [Sex: female Subcomponent: liver] &  &  &  &  &  &  & LIBI & Biota lipid concentrations & NA & NA\\
\hline
21032 & IPMMCF33 & Concentration of indeno(1,2,3-cd)pyrene \{CAS 193-39-5\} per unit wet weight of biota \{Clupea harengus (ITIS: 161722: WoRMS 126417) [Subcomponent: liver]\} & S0600045 & Concentration &  &  & CS002118 & indeno(1,2,3-cd)pyrene &  & 193-39-5 & S055 & per unit wet weight of & MAT01963 & biota & BE006619 & Clupea harengus (ITIS: 161722: WoRMS 126417) [Subcomponent: liver] &  &  &  &  &  &  & BCAH & Concentration of polycyclic aromatic hydrocarbons (PAHs) in biota & NA & NA\\
\hline
43684 & ZNCNPEXT & Concentration of zinc \{Zn CAS 7440-66-6\} per unit dry weight of sediment by acid digestion and inductively-coupled plasma atomic emission spectroscopy & S0600045 & Concentration &  &  & CS002678 & zinc & Zn & 7440-66-6 & S041 & per unit dry weight of & MAT00136 & sediment &  &  & S0360 & acid digestion & S04235 & inductively-coupled plasma atomic emission spectroscopy &  &  & MTSD & Metal concentrations in sediment & NA & NA\\
\hline
18050 & IC001518 & Concentration of 2,4'-dichlorodiphenyldichloroethane \{o,p'-DDD mitotane CAS 53-19-0\} per unit wet weight of biota \{Scomber scombrus (ITIS: 172414: WoRMS 127023) [Subcomponent: muscle tissue]\} & S0600045 & Concentration &  &  & CS001103 & 2,4'-dichlorodiphenyldichloroethane & o,p'-DDD mitotane & 53-19-0 & S055 & per unit wet weight of & MAT01963 & biota & BE007119 & Scomber scombrus (ITIS: 172414: WoRMS 127023) [Subcomponent: muscle tissue] &  &  &  &  &  &  & PEBI & Pesticide concentrations in biota & NA & NA\\
\hline
41001 & Z9A09040 & Length (from the front of the eye to the tip of the telson) of Thysanoessa inermis (ITIS: 95573: WoRMS 110708) [Stage: juvenile+sub-adult Sex: female] & S0600096 & Length (from the front of the eye to the tip of the telson) &  &  &  &  &  &  & S029 & not applicable & MAT00906 & not applicable & BE006101 & Thysanoessa inermis (ITIS: 95573: WoRMS 110708) [Stage: juvenile+sub-adult Sex: female] &  &  &  &  &  &  & BLEN & Zooplankton and zoobenthos morphological parameters & NA & NA\\
\hline
23500 & MUM13104 & Lipid-normalised concentration of 2,4',5-trichlorobiphenyl \{PCB31 CAS 16606-02-3\} in biota \{Platichthys flesus (ITIS: 172894: WoRMS 127141) [Subcomponent: liver]\} & S0600044 & Lipid-normalised concentration &  &  & CS001880 & 2,4',5-trichlorobiphenyl & PCB31 & 16606-02-3 & S026 & in & MAT01963 & biota & BE006615 & Platichthys flesus (ITIS: 172894: WoRMS 127141) [Subcomponent: liver] &  &  &  &  &  &  & BCPB & Concentration of polychlorobiphenyls (PCBs) in biota & NA & NA\\
\hline
33491 & SNCONHMP & Normalised nominal concentration of humic acid in peat by drying, grinding, NaOH extraction and filtration and colorimetric analysis and computation of Blundell and Barber (2005) &  &  &  &  &  &  &  &  & S026 & in & MAT00913 & peat &  &  &  &  &  &  &  &  & STOM & Concentration of organic matter in sediments & NA & NA\\
\hline
40715 & Z39CO01K & Abundance of Eurytemora affinis (ITIS: 85863: WoRMS 104872) [Stage: copepodites C1-C3] per unit volume of the water body by optical microscopy & S0600002 & Abundance &  &  &  &  &  &  & S053 & per unit volume of the & MAT00640 & water body & BE002211 & Eurytemora affinis (ITIS: 85863: WoRMS 104872) [Stage: copepodites C1-C3] &  &  & S0423 & optical microscopy &  &  & ZATX & Zooplankton taxonomy-related abundance per unit volume of the water column & NA & NA\\
\hline
\end{tabular}
\end{table}

\hypertarget{gebruik-van-de-koppelingstabel}{%
\section{Gebruik van de koppelingstabel}\label{gebruik-van-de-koppelingstabel}}

Voor veel combinaties van AQUO (um-aquo) velden zijn in het verleden P01 parameternamen gezocht. Deels staan deze in de mappingP01 tabel die onderhouden wordt door AQUO. Deels staan deze in andere locale bestanden die ad hoc mappings hebben gefaciliteerd (bronnen: RWS, MARIS, Deltares). In dit project hebben we in eerste instantie geprobeerd om AQUO codes te koppelen aan P01 namen. In een tweede stap worden de P01 namen opgesplitst in hun semantische onderdelen, en deze zullen dan gematcht worden met de afzonderlijke AQUO namen in de daarvoor geschikte velden. Hiervoor is het nodig de verschillende ``S'' velden te koppelen aan AQUO tabellen. De waarden in deze tabellen worden dan gekoppeld met behulp van tabel \ref{tab:p01s-tabel}.

Op 25 mei 2022 zijn de volgende ``S'' tabellen gevonden

\begin{table}

\caption{\label{tab:unnamed-chunk-2}Beschrijving van tabellen met semantische componenten voor de P01 beschrijvende parameters. Deze versie is op 25 mei 2022 van de [seadata website](https://vocab.seadatanet.org/search) opgehaaald. }
\centering
\begin{tabular}[t]{l|l|l|r|r|l}
\hline
Library & Title & Alt.Title & Version & Members & Modified\\
\hline
S02 & BODC parameter semantic model relationships between what theme and where theme & Where/what relationships & 18 & 58 & 5/18/2022 4:00\\
\hline
S03 & BODC parameter semantic model sample preparation entity descriptions & Sample preparation & 76 & 290 & 5/14/2022 4:00\\
\hline
S04 & BODC parameter semantic model analytical method entity descriptions & Analytical method & 137 & 562 & 4/28/2022 4:00\\
\hline
S05 & BODC parameter semantic model data processing entity descriptions & Data processing & 55 & 242 & 4/28/2022 4:00\\
\hline
S06 & BODC parameter semantic model parameter entity names & Parameter entity names & 118 & 273 & 5/24/2022 4:00\\
\hline
S07 & BODC parameter semantic model parameter statistic & Parameter statistic & 14 & 49 & 4/28/2022 4:00\\
\hline
S11 & BODC parameter semantic model biological entity development stage terms & Biological entity life stage terms & 30 & 97 & 4/28/2022 4:00\\
\hline
S21 & BODC parameter semantic model sphere names & Sphere names & 12 & 22 & 2/6/2015 2:00\\
\hline
S23 & BODC semantic model sphere phase names & Sphere phase names & 7 & 15 & 11/1/2019 2:00\\
\hline
S25 & BODC parameter semantic model biological entity names & Biological entity names & 150 & 8510 & 5/24/2022 4:00\\
\hline
S26 & BODC parameter semantic model matrices & BODC matrices & 35 & 207 & 3/30/2022 4:00\\
\hline
S27 & BODC parameter semantic model chemical substances & BODC substances & 147 & 1962 & 4/7/2022 4:00\\
\hline
\end{tabular}
\end{table}

\hypertarget{mapping-van-wat-tabellen}{%
\chapter{Mapping van ``wat'' tabellen}\label{mapping-van-wat-tabellen}}

\hypertarget{vergelijking-ddl-metadata-en-p01-vocabulaire}{%
\section{Vergelijking DDL metadata en P01 vocabulaire}\label{vergelijking-ddl-metadata-en-p01-vocabulaire}}

De Data Distributie Laag (DDL) bevat fysische, chemische en deels biologische gegevens van Rijkswaterstaat in het waterdomein. Zie voor een overzicht \href{https://rijkswaterstaatdata.nl/waterdata/}{deze weblink}. Hier staan ook links naar instructies voor benadering van de webservices. Een beknopt overzicht van de DDL kan worden opgevraagd via de metadata service. Hieruit blijkt dat het totale aantal unieke combinaties van de metadatavelden Compartiment.Code, Grootheid.Code, Hoedanigheid.Code, Parameter.Code in de DDL is 1710.

Binnen de aquo domeintabellen is in eerdere projecten de mapping tabel idsw\_aquo\_map\_PO1 gemaakt om sommige elementen van aquo te koppelen aan de seadatanet semantiek. Deze tabel wordt hier als uitgangspunt gebruikt om P01 termen te koppelen aan metadata uit de Data Distributielaag (DDL). Hierna moeten de P01 termen uitgesplitst worden in de verschillende onderliggende componenten om een 1:1 mapping te verkrijgen van de individuele \emph{S}-termen.

De AquoMetadataLijst uit de DDL metadata is grotendeels opgebouwd volgens AQUO IM-metingen, maar wijkt op een aantal punten af. Onder andere doordat het veld Hoedanigheid.code is opgebouwd uit twee verschillende AQUO codes, namelijk Hoedanigheid.code en Monstercriterium.code. Idem voor de omschrijvingen. In de idsw\_aquo\_map\_PO1 zijn deze twee velden niet gecombineerd. Voor een kansrijke automatische mapping worden in idsw\_aquo\_map\_PO1 deze velden aan elkaar geplakt. In idsw\_aquo\_map\_PO1 zijn ``lege'' velden echt leeg terwijl deze in de DDL metadata gevuld zijn met ``NVT''. Dus, vóórdat de P01 koppeltabel gekoppeld kan worden aan de AQUO catalogustabel worden Hoedanigheid.code en Monstercriterium.code gecombineerd in één veld en alle lege plekken gevuld met ``NVT''.

De metadata uit de DDL worden hierna gecombineerd met de mapping\_P01 tabel op de velden:

waarin \emph{Aquo\_Monstercriterium\_Hoedanigheid\_code} de combinatie is van monstercriterium en hoedanigheid.

De uiteindelijke mapping geeft als resultaat een tabel met 2015 regels.

\hypertarget{uitsplitsing-in-onderdelen-van-p01}{%
\section{Uitsplitsing in onderdelen van P01}\label{uitsplitsing-in-onderdelen-van-p01}}

\begin{table}

\caption{\label{tab:unnamed-chunk-5}BODC "S-tabellen" die onderdelen bevatten van de P01 parameter tabel.}
\centering
\fontsize{10}{12}\selectfont
\begin{tabu} to \linewidth {>{\raggedright}X>{\raggedright}X>{\raggedright}X}
\hline
Library & Alt.Title & Title\\
\hline
S02 & Where/what relationships & BODC parameter semantic model relationships between what theme and where theme\\
\hline
S03 & Sample preparation & BODC parameter semantic model sample preparation entity descriptions\\
\hline
S04 & Analytical method & BODC parameter semantic model analytical method entity descriptions\\
\hline
S05 & Data processing & BODC parameter semantic model data processing entity descriptions\\
\hline
S06 & Parameter entity names & BODC parameter semantic model parameter entity names\\
\hline
S07 & Parameter statistic & BODC parameter semantic model parameter statistic\\
\hline
S11 & Biological entity life stage terms & BODC parameter semantic model biological entity development stage terms\\
\hline
S21 & Sphere names & BODC parameter semantic model sphere names\\
\hline
S23 & Sphere phase names & BODC semantic model sphere phase names\\
\hline
S25 & Biological entity names & BODC parameter semantic model biological entity names\\
\hline
S26 & BODC matrices & BODC parameter semantic model matrices\\
\hline
S27 & BODC substances & BODC parameter semantic model chemical substances\\
\hline
\end{tabu}
\end{table}

\hypertarget{parameter-vs-bodc-substances}{%
\subsection{parameter vs BODC substances}\label{parameter-vs-bodc-substances}}

!!! Een eenduidige mapping kan onafhankelijk van eerdere mappings worden gemaakt door het vergelijken van CAS nummers, deze staan in p01s tabel, maar het staat verborgen in de P01 beschrijving. Een betere manier is waarschijnlijk om de S27 termen te relateren aan CAS nummers door deze op te halen via \url{https://www.w3.org/2002/07/owl\#sameAs} , bijvoorbeeld:

S27 \url{http://vocab.nerc.ac.uk/collection/S27/current/CS026903/}
\url{https://www.w3.org/2002/07/owl\#sameAs}
\url{https://chem.nlm.nih.gov/chemidplus/rn/17181-37-2}

zo is voor Silicate een van de bijbehorende links ``same as'' naar \url{https://chem.nlm.nih.gov/chemidplus/rn/17181-37-2}. Hierin staat een CAS nummer. Door deze te relateren kan een verbinding worden gemaakt met de technische tabel ``chemische stof''.

Vanuit de gekoppelde tabel wordt een mapping geëxtraheerd door de unieke combinaties van relevante AQUO (parameter.code, parameter.omschrijving) en SDN (S27\_preflabel, S27\_altlabel, s27\_conceptid) velden te combineren. Deze lijst dient nog handmatig gecontroleerd te worden. Met deze methode kunnen 161 van de 875 (totaal aantal unieke parameter.code waarden) gecombineerd worden met een S27 term.

\begin{tabular}[t]{l|l|l|l|l}
\hline
Parameter.Code & Parameter.Omschrijving & s27\_conceptid & S27\_preflabel & S27\_altlabel\\
\hline
O2 & zuurstof & CS002779 & oxygen & O2\\
\hline
Fe & ijzer & CS000284 & total iron & total\_Fe\\
\hline
Ca & calcium & CS002921 & calcium & Ca\\
\hline
K & kalium & CS002923 & potassium & K\\
\hline
Mg & magnesium & CS002981 & magnesium & Mg\\
\hline
Na & natrium & CS002904 & sodium & Na\\
\hline
DC4ySn & dibutyltin (kation) & CS000431 & dibutyltin & DBT dibutylstannane\\
\hline
DFySn & difenyltin (kation) & CS000452 & diphenyltin & DPT diphenylstannane\\
\hline
TC4ySn & tributyltin (kation) & CS002650 & tributyltin cation & tributylstannylium TBT+\\
\hline
T4C4ySn & tetrabutyltin & CS003148 & tetrabutyltin & \\
\hline
Ag & zilver & CS002615 & silver & Ag\\
\hline
As & arseen & CS002328 & arsenic & As\\
\hline
B & boor & CS003263 & boron & B\\
\hline
Ba & barium & CS002335 & barium & Ba\\
\hline
Be & beryllium & CS003266 & beryllium & Be\\
\hline
Cd & cadmium & CS002363 & cadmium & Cd\\
\hline
Co & kobalt & CS002447 & cobalt & Co\\
\hline
Cr & chroom & CS002377 & chromium & Cr\\
\hline
Cu & koper & CS002454 & copper & Cu\\
\hline
Hg & kwik & CS001621 & total mercury & total\_Hg\\
\hline
Li & lithium & CS002552 & lithium & Li\\
\hline
Mn & mangaan & CS000305 & total manganese & total\_Mn\\
\hline
Mo & molybdeen & CS003003 & molybdenum & Mo\\
\hline
Ni & nikkel & CS002566 & nickel & Ni\\
\hline
Pb & lood & CS002545 & lead & Pb\\
\hline
Rb & rubidium & CS003116 & rubidium & Rb\\
\hline
Sb & antimoon & CS002962 & antimony & Sb\\
\hline
Se & seleen & CS002608 & selenium & Se\\
\hline
Sn & tin & CS003004 & tin & Sn\\
\hline
Sr & strontium & CS002622 & strontium & Sr\\
\hline
Te & tellurium & CS003145 & tellurium & Te\\
\hline
Ti & titaan & CS002918 & titanium & Ti\\
\hline
Tl & thallium & CS003132 & thallium & Tl\\
\hline
U & uranium & CS026907 & uranium & U\\
\hline
V & vanadium & CS002664 & vanadium & V\\
\hline
Zn & zink & CS002678 & zinc & Zn\\
\hline
aedsfn & alfa-endosulfan & CS002307 & alpha-endosulfan & \\
\hline
aHCH & alfa-hexachloorcyclohexaan & CS002321 & alpha-hexachlorocyclohexane & alpha-HCH\\
\hline
alCl & alachloor & CS001173 & alachlor & \\
\hline
aldn & aldrin & CS002293 & aldrin & \\
\hline
Ant & antraceen & CS002027 & anthracene & \\
\hline
atzne & atrazine & CS001579 & atrazine & \\
\hline
BaP & benzo(a)pyreen & CS001908 & benzo(a)pyrene & \\
\hline
BbF & benzo(b)fluorantheen & CS001915 & benzo(b)fluoranthene & \\
\hline
bedsfn & beta-endosulfan & CS003282 & beta-endosulfan & \\
\hline
Ben & benzeen & CS003272 & benzene & \\
\hline
bentzn & bentazon & CS001187 & bentazone & bentazon\\
\hline
BghiPe & benzo(ghi)peryleen & CS001936 & benzo(g,h,i)perylene & \\
\hline
bHCH & beta-hexachloorcyclohexaan & CS002349 & beta-hexachlorocyclohexane & beta-HCH\\
\hline
BkF & benzo(k)fluorantheen & CS001950 & benzo(k)fluoranthene & \\
\hline
captn & captan & CS003284 & captan & \\
\hline
cHCH & gamma-hexachloorcyclohexaan (lindaan) & CS002503 & gamma-hexachlorocyclohexane & lindane gamma-HCH\\
\hline
CHLFa & chlorofyl-a & CS002942 & chlorophyllide-a & \\
\hline
cHpClepO & cis-heptachloorepoxide & CS002440 & cis-heptachlor epoxide & \\
\hline
ClBen & chloorbenzeen & CS003286 & chlorobenzene & \\
\hline
Clfvfs & chloorfenvinfos & CS001201 & chlorfenvinphos & chlorfenvinfos\\
\hline
Clidzn & chloridazon & CS001208 & chloridazon & \\
\hline
Clpfm & chloorprofam & CS001600 & chlorpropham & \\
\hline
Cltlrn & chloortoluron & CS001222 & chlortoluron & chlorotoluron\\
\hline
cumfs & cumafos & CS000375 & coumaphos & \\
\hline
cumn & cumeen & CS003258 & (1-methylethyl)benzene & isopropylbenzene cumene\\
\hline
C1yazfs & methylazinfos & CS001586 & azinphos-methyl & \\
\hline
C1yClprfs & methylchloorpyrifos & CS001215 & chlorpyrifos-methyl & \\
\hline
C1ymsfrn & methyl-metsulfuron & CS000725 & metsulfuron-methyl & \\
\hline
C1yprmfs & methylpirimifos & CS001425 & pirimiphos-methyl & \\
\hline
C1yprton & methylparathion & CS003164 & parathion-methyl & \\
\hline
c12DClC2e & cis-1,2-dichlooretheen & CS000333 & cis-1,2-dichloroethene & \\
\hline
C2yazfs & ethylazinfos & CS001180 & azinphos-ethyl & \\
\hline
C2yBen & ethylbenzeen & CS003317 & ethylbenzene & \\
\hline
C2yClprfs & ethylchloorpyrifos & CS003289 & chlorpyrifos & \\
\hline
C2yprton & ethylparathion & CS000781 & parathion-ethyl & \\
\hline
Daznn & diazinon & CS000424 & diazinon & \\
\hline
DClC1a & dichloormethaan & CS003270 & dichloromethane & \\
\hline
DClvs & dichloorvos & CS003283 & dichlorvos & \\
\hline
DEHP & bis(2-ethylhexyl)ftalaat (DEHP) & CS001194 & bis(2-ethylhexyl)phthalate & DEHP\\
\hline
dHCH & delta-hexachloorcyclohexaan & CS000396 & delta-hexachlorocyclohexane & delta-HCH\\
\hline
dieldn & dieldrin & CS002468 & dieldrin & \\
\hline
Dmtat & dimethoaat & CS000438 & dimethoate & \\
\hline
dmtn & deltamethrin & CS000403 & deltamethrin & \\
\hline
dodne & dodine & CS000508 & dodine & 1-dodecylguanidine acetate\\
\hline
Dtann & dithianon & CS000459 & dithianon & \\
\hline
Durn & diuron & CS003293 & diuron & 3-(3,4-dichlorophenyl)-1,1-dimethylurea\\
\hline
endn & endrin & CS002489 & endrin & \\
\hline
esfvlrt & esfenvaleraat & CS000543 & esfenvalerate & \\
\hline
fenamfs & fenamifos & CS000564 & fenamiphos & \\
\hline
fenOxcb & fenoxycarb & CS000578 & fenoxycarb & \\
\hline
feNO2ton & fenitrothion & CS000571 & fenitrothion & \\
\hline
fenton & fenthion & CS000585 & fenthion & \\
\hline
Flu & fluorantheen & CS002069 & fluoranthene & \\
\hline
HCB & hexachloorbenzeen & CS002531 & hexachlorobenzene & HCB\\
\hline
heptnfs & heptenofos & CS000641 & heptenophos & \\
\hline
HpCl & heptachloor & CS002510 & heptachlor & \\
\hline
HxClbtDen & hexachloorbutadieen & CS002524 & hexachloro-1,3-butadiene & \\
\hline
HxClC2a & hexachloorethaan & CS000648 & hexachloroethane & \\
\hline
idn & isodrin & CS002538 & isodrin & \\
\hline
imdcpd & imidacloprid & CS000683 & imidacloprid & \\
\hline
InP & indeno(1,2,3-cd)pyreen & CS002118 & indeno(1,2,3-cd)pyrene & \\
\hline
iptrn & isoproturon & CS000690 & isoproturon & \\
\hline
irgrl & irgarol & CS003184 & irgarol & \\
\hline
lcyhltn & lambda-cyhalothrin & CS003187 & lambda-cyhalothrin & \\
\hline
linrn & linuron & CS000697 & linuron & \\
\hline
malton & malathion & CS001348 & malathion & \\
\hline
metbtazrn & metabenzthiazuron & CS000704 & methabenzthiazuron & \\
\hline
metlCl & metolachloor & CS001362 & metolachlor & \\
\hline
mevfs & mevinfos & CS000732 & mevinphos & \\
\hline
Mlnrn & monolinuron & CS000739 & monolinuron & \\
\hline
Naf & naftaleen & CS002013 & naphthalene & \\
\hline
PBDE100 & 2,2',4,4',6-pentabroomdifenylether & CS002146 & 2,2',4,4',6-pentabromodiphenyl ether & BDE100\\
\hline
PBDE138 & 2,2',3,4,4',5'-hexabroomdifenylether & CS002132 & 2,2',3,4,4',5'-hexabromodiphenyl ether & BDE138\\
\hline
PBDE153 & 2,2',4,4',5,5'-hexabroomdifenylether & CS000998 & 2,2',4,4',5,5'-hexabromodiphenyl ether & BDE153\\
\hline
PBDE154 & 2,2',4,4',5,6'-hexabroomdifenylether & CS001005 & 2,2',4,4',5,6'-hexabromodiphenyl ether & BDE154\\
\hline
PBDE209 & 2,2',3,3',4,4',5,5',6,6'-decabroomdiphenylether & CS000382 & decabromodiphenyl ether & BDE209 decabromodiphenyl oxide\\
\hline
PBDE28 & 2,4,4'-tribroomdifenylether & CS002209 & 2,4,4'-tribromodiphenyl ether & BDE28\\
\hline
PBDE47 & 2,2',4,4'-tetrabroomdifenylether & CS001026 & 2,2',4,4'-tetrabromodiphenyl ether & BDE47\\
\hline
PBDE49 & 2,2',4,5'-tetrabroomdifenylether & CS001040 & 2,2',4,5'-tetrabromodiphenyl ether & BDE49\\
\hline
PBDE85 & 2,2',3,4,4'-pentabroomdifenylether & CS002139 & 2,2',3,4,4'-pentabromodiphenyl ether & BDE85\\
\hline
PBDE99 & 2,2',4,4',5-pentabroomdifenylether & CS001012 & 2,2',4,4',5-pentabromodiphenyl ether & BDE99\\
\hline
PeClBen & pentachloorbenzeen & CS002587 & pentachlorobenzene & \\
\hline
PeClFol & pentachloorfenol & CS003165 & pentachlorophenol & \\
\hline
pirmcb & pirimicarb & CS000809 & pirimicarb & \\
\hline
propxr & propoxur & CS001432 & propoxur & \\
\hline
pyrdbn & pyridaben & CS000830 & pyridaben & \\
\hline
pyrpxfn & pyriproxyfen & CS001439 & pyriproxyfen & \\
\hline
simzne & simazine & CS000837 & simazine & \\
\hline
styrn & styreen & CS003166 & styrene & \\
\hline
s4C9yFol & som 4-nonylfenol-isomeren (vertakt) &  &  & \\
\hline
Tazfs & triazofos & CS001495 & triazophos & \\
\hline
TClC1a & trichloormethaan (chloroform) & CS003033 & trichloromethane & chloroform\\
\hline
TClC2e & trichlooretheen (tri) & CS003226 & 1,1,2-trichloroethene & trichloroethylene TCE\\
\hline
tefbzrn & teflubenzuron & CS001446 & teflubenzuron & \\
\hline
terbtn & terbutrin & CS000851 & terbutryn & \\
\hline
terC4yazne & terbutylazine & CS000858 & terbuthylazine & TBA\\
\hline
Tfrlne & trifluraline & CS001243 & trifluralin & \\
\hline
TFySn & trifenyltin (kation) & CS001250 & triphenyltin cation & TPT+\\
\hline
tHpClepO & trans-heptachloorepoxide & CS002636 & trans-heptachlor epoxide & \\
\hline
Tol & tolueen & CS003183 & methylbenzene & toluene\\
\hline
tolcfsC1y & tolclofos-methyl & CS000893 & tolclofos-methyl & \\
\hline
t12DClC2e & trans-1,2-dichlooretheen & CS003149 & trans-1,2-dichloroethene & \\
\hline
T4ClC1a & tetrachloormethaan (tetra) & CS003013 & tetrachloromethane & CCl4 carbon tetrachloride\\
\hline
T4ClC2e & tetrachlooretheen (per) & CS003138 & tetrachloroethene & tetrachloroethylene\\
\hline
11DClC2a & 1,1-dichloorethaan & CS000942 & 1,1-dichloroethane & \\
\hline
11DClC2e & 1,1-dichlooretheen & CS003250 & 1,1-dichloroethene & \\
\hline
111TClC2a & 1,1,1-trichloorethaan & CS003246 & 1,1,1-trichloroethane & \\
\hline
112TClC2a & 1,1,2-trichloorethaan & CS003233 & 1,1,2-trichloroethane & \\
\hline
1122T4ClC2a & 1,1,2,2-tetrachloorethaan & CS003232 & 1,1,2,2-tetrachloroethane & \\
\hline
12DClBen & 1,2-dichloorbenzeen & CS003235 & 1,2-dichlorobenzene & \\
\hline
12DClC2a & 1,2-dichloorethaan & CS003247 & 1,2-dichloroethane & \\
\hline
12DClC3a & 1,2-dichloorpropaan & CS001516 & 1,2-dichloropropane & \\
\hline
12xyln & 1,2-xyleen & CS000963 & 1,2-dimethylbenzene & 1,2-xylene o-xylene\\
\hline
123TClBen & 1,2,3-trichloorbenzeen & CS000949 & 1,2,3-trichlorobenzene & 1,2,3-TCB\\
\hline
124TClBen & 1,2,4-trichloorbenzeen & CS000956 & 1,2,4-trichlorobenzene & 1,2,4-TCB\\
\hline
13DClBen & 1,3-dichloorbenzeen & CS003249 & 1,3-dichlorobenzene & \\
\hline
135TClBen & 1,3,5-trichloorbenzeen & CS000970 & 1,3,5-trichlorobenzene & 1,3,5-TCB\\
\hline
14DClBen & 1,4-dichloorbenzeen & CS003234 & 1,4-dichlorobenzene & \\
\hline
24DDT & 2,4'-dichloordifenyltrichloorethaan & CS001117 & 2,4'-dichlorodiphenyltrichloroethane & o,p'-DDT\\
\hline
24DP & 2,4-dichloorfenoxypropionzuur & CS001124 & dichloroprop & 2,4-DP 2,4-dichlorophenoxypropionic acid\\
\hline
44DDD & 4,4'-dichloordifenyldichloorethaan & CS002244 & 4,4'-dichlorodiphenyldichloroethane & p,p'-DDD\\
\hline
44DDE & 4,4'-dichloordifenyldichlooretheen & CS002251 & 4,4'-dichlorodiphenyldichloroethylene & p,p'-DDE\\
\hline
44DDT & 4,4'-dichloordifenyltrichloorethaan & CS002258 & 4,4'-dichlorodiphenyltrichloroethane & p,p'-DDT\\
\hline
BaP & benzo(a)pyreen & CS001929 & benzo(e)pyrene & \\
\hline
teldn & telodrin & CS002629 & telodrin & isobenzan\\
\hline
\end{tabular}

\hypertarget{compartimenten-vs-s26-bodc-matrices}{%
\subsection{Compartimenten vs S26 BODC matrices}\label{compartimenten-vs-s26-bodc-matrices}}

\begin{tabular}[t]{l|l|l|l|l}
\hline
Compartiment.Code & Compartiment.Omschrijving & Hoedanigheid.Code & S26\_conceptid & S26\_preflabel\\
\hline
BS & Bodem/Sediment & NVT & MAT00850 & bed\\
\hline
OW & Oppervlaktewater & NVT & MAT00633 & water body [dissolved plus reactive particulate phase]\\
\hline
OW & Oppervlaktewater & NVT & MAT00640 & water body\\
\hline
OW & Oppervlaktewater & nf & MAT00626 & water body [dissolved plus reactive particulate <unknown phase]\\
\hline
OW & Oppervlaktewater & NVT & MAT00997 & water body [particulate >GF/F phase]\\
\hline
ZS & Zwevende stof & dg & MAT00528 & suspended particulate material\\
\hline
\end{tabular}

\hypertarget{grootheden-vs-s06-parameter-entity-names}{%
\subsection{Grootheden vs S06 Parameter entity names}\label{grootheden-vs-s06-parameter-entity-names}}

\begin{tabular}[t]{l|l|l|l|l}
\hline
Grootheid.Code & Grootheid.Omschrijving & Hoedanigheid.Code & S06\_conceptid & S06\_preflabel\\
\hline
AANTPOPVTE & Aantal per oppervlakte & NVT & S0600002 & Abundance\\
\hline
VERZDGGD & Verzadigingsgraad & NVT & S0600045 & Concentration\\
\hline
SALNTT & Saliniteit & NVT & S0600085 & Salinity\\
\hline
CONCTTE & (massa)Concentratie & nf & S0600045 & Concentration\\
\hline
CONCTTE & (massa)Concentratie & NVT & S0600045 & Concentration\\
\hline
T & Temperatuur & NVT & S0600082 & Temperature\\
\hline
CONCTTE & (massa)Concentratie & NVT &  & \\
\hline
MASSFTE & Massafractie & dg & S0600045 & Concentration\\
\hline
\end{tabular}

\hypertarget{specials---grain-size}{%
\subsection{Specials - Grain size}\label{specials---grain-size}}

\textbf{in water}

\begin{tabular}[t]{r|l|l|l|l|l|l|l|l|l|l|l|l|l|l|l|l|l|l|l|l|l|l|l|l|l|l}
\hline
n\_code & P01\_conceptid & P01\_preflabel & S06\_conceptid & S06\_preflabel & S07\_conceptid & S07\_preflabel & s27\_conceptid & S27\_preflabel & S27\_altlabel & CAS no & S02\_conceptid & S02\_preflabel & S26\_conceptid & S26\_preflabel & S25\_conceptid & S25\_preflabel & S03\_conceptid & S03\_preflabel & S04\_conceptid & S04\_preflabel & S05\_conceptid & S05\_preflabel & P02\_conceptid & P02\_preflabel & S21\_conceptid & S21\_preflabel\\
\hline
23075 & MNGSIXAG & Grain-size mean of aggregates in the water body by image analysis & S0600192 & Grain-size & S0700003 & mean &  &  &  &  & S028 & in the & MAT00640 & water body &  &  &  &  & S0415 & image analysis &  &  & SPGS & Suspended particulate material grain size parameters & NA & NA\\
\hline
23239 & MSAGIXPZ & Grain-size median of aggregates in the water body by image analysis & S0600192 & Grain-size & S0700004 & median &  &  &  &  & S028 & in the & MAT00640 & water body &  &  &  &  & S0415 & image analysis &  &  & ABAG & Suspended particulate material aggregates & NA & NA\\
\hline
\end{tabular}

\textbf{in sediment}

\begin{tabular}[t]{r|l|l|l|l|l|l|l|l|l|l|l|l|l|l|l|l|l|l|l|l|l|l|l|l|l|l}
\hline
n\_code & P01\_conceptid & P01\_preflabel & S06\_conceptid & S06\_preflabel & S07\_conceptid & S07\_preflabel & s27\_conceptid & S27\_preflabel & S27\_altlabel & CAS no & S02\_conceptid & S02\_preflabel & S26\_conceptid & S26\_preflabel & S25\_conceptid & S25\_preflabel & S03\_conceptid & S03\_preflabel & S04\_conceptid & S04\_preflabel & S05\_conceptid & S05\_preflabel & P02\_conceptid & P02\_preflabel & S21\_conceptid & S21\_preflabel\\
\hline
15531 & GRSIZEMM & Grain-size minimum of particles in sediment by visual estimation & S0600192 & Grain-size & S0700005 & minimum &  &  &  &  & S026 & in & MAT00136 & sediment &  &  &  &  & S04366 & visual estimation &  &  & MNGS & Sediment grain size parameters & NA & NA\\
\hline
15532 & GRSIZEMX & Grain-size maximum of particles in sediment by visual estimation & S0600192 & Grain-size & S0700002 & maximum &  &  &  &  & S026 & in & MAT00136 & sediment &  &  &  &  & S04366 & visual estimation &  &  & MNGS & Sediment grain size parameters & NA & NA\\
\hline
21902 & KRGSPSXX & Grain-size kurtosis of particles in sediment by particle sizer & S0600192 & Grain-size & S0700030 & kurtosis &  &  &  &  & S026 & in & MAT00136 & sediment &  &  &  &  & S04278 & particle sizer &  &  & MNGS & Sediment grain size parameters & NA & NA\\
\hline
21903 & KRGSSSXX & Grain-size kurtosis of particles in sediment by sieving and settling tube method & S0600192 & Grain-size & S0700030 & kurtosis &  &  &  &  & S026 & in & MAT00136 & sediment &  &  &  &  & S04321 & sieving and settling tube method &  &  & MNGS & Sediment grain size parameters & NA & NA\\
\hline
21905 & KRTSSSXX & Grain-size kurtosis of particles in sediment by sieving and settling tube method & S0600192 & Grain-size & S0700030 & kurtosis &  &  &  &  & S026 & in & MAT00136 & sediment &  &  &  &  & S04321 & sieving and settling tube method &  &  & MNGS & Sediment grain size parameters & NA & NA\\
\hline
22696 & MDGSPPXX & Grain-size median of particles in sediment by optical microscopy (coarse fraction) and pipette method (fines) & S0600192 & Grain-size & S0700004 & median &  &  &  &  & S026 & in & MAT00136 & sediment &  &  &  &  & S04275 & optical microscopy (coarse fraction) and pipette method (fines) &  &  & MNGS & Sediment grain size parameters & NA & NA\\
\hline
22698 & MDGSPSXX & Grain-size median of particles in sediment by particle sizer & S0600192 & Grain-size & S0700004 & median &  &  &  &  & S026 & in & MAT00136 & sediment &  &  &  &  & S04278 & particle sizer &  &  & MNGS & Sediment grain size parameters & NA & NA\\
\hline
22699 & MDGSSSXX & Grain-size median of particles in sediment by sieving and settling tube method & S0600192 & Grain-size & S0700004 & median &  &  &  &  & S026 & in & MAT00136 & sediment &  &  &  &  & S04321 & sieving and settling tube method &  &  & MNGS & Sediment grain size parameters & NA & NA\\
\hline
23076 & MNGSPSNC & Grain-size mean of particles in sediment [non-carbonate phase] by acidification and particle sizer & S0600192 & Grain-size & S0700003 & mean &  &  &  &  & S026 & in & MAT00129 & sediment [non-carbonate phase] &  &  & S0366 & acidification & S04278 & particle sizer &  &  & MNGS & Sediment grain size parameters & NA & NA\\
\hline
23078 & MNGSPSSA & Grain-size mean of particles in sediment 63-1000um by particle sizer & S0600192 & Grain-size & S0700003 & mean &  &  &  &  & S026 & in & MAT00059 & sediment 63-1000um &  &  &  &  & S04278 & particle sizer &  &  & MNGS & Sediment grain size parameters & NA & NA\\
\hline
23079 & MNGSPSXX & Grain-size mean of particles in sediment by particle sizer & S0600192 & Grain-size & S0700003 & mean &  &  &  &  & S026 & in & MAT00136 & sediment &  &  &  &  & S04278 & particle sizer &  &  & MNGS & Sediment grain size parameters & NA & NA\\
\hline
23081 & MNGSSSXX & Grain-size mean of particles in sediment by sieving and settling tube method & S0600192 & Grain-size & S0700003 & mean &  &  &  &  & S026 & in & MAT00136 & sediment &  &  &  &  & S04321 & sieving and settling tube method &  &  & MNGS & Sediment grain size parameters & NA & NA\\
\hline
23133 & MOGSPSXX & Grain-size mode of particles in sediment by particle sizer & S0600192 & Grain-size & S0700041 & mode &  &  &  &  & S026 & in & MAT00136 & sediment &  &  &  &  & S04278 & particle sizer &  &  & MNGS & Sediment grain size parameters & NA & NA\\
\hline
23134 & MOGSSSXX & Grain-size mode of particles in sediment by sieving and settling tube method & S0600192 & Grain-size & S0700041 & mode &  &  &  &  & S026 & in & MAT00136 & sediment &  &  &  &  & S04321 & sieving and settling tube method &  &  & MNGS & Sediment grain size parameters & NA & NA\\
\hline
27891 & PC01SSXX & Grain-size 1st percentile of particles in sediment by sieving and settling tube method and analysis of cumulative frequency ordered by phi (coarse to fine) & S0600192 & Grain-size & S0700039 & 1st percentile &  &  &  &  & S026 & in & MAT00136 & sediment &  &  &  &  & S04321 & sieving and settling tube method & S050063 & analysis of cumulative frequency ordered by phi (coarse to fine) & MNGS & Sediment grain size parameters & NA & NA\\
\hline
27893 & PC05PSXX & Grain-size 5th percentile of particles in sediment by particle sizer and analysis of cumulative frequency ordered by phi (coarse to fine) & S0600192 & Grain-size & S0700034 & 5th percentile &  &  &  &  & S026 & in & MAT00136 & sediment &  &  &  &  & S04278 & particle sizer & S050063 & analysis of cumulative frequency ordered by phi (coarse to fine) & MNGS & Sediment grain size parameters & NA & NA\\
\hline
27894 & PC05SSXX & Grain-size 5th percentile of particles in sediment by sieving and settling tube method and analysis of cumulative frequency ordered by phi (coarse to fine) & S0600192 & Grain-size & S0700034 & 5th percentile &  &  &  &  & S026 & in & MAT00136 & sediment &  &  &  &  & S04321 & sieving and settling tube method & S050063 & analysis of cumulative frequency ordered by phi (coarse to fine) & MNGS & Sediment grain size parameters & NA & NA\\
\hline
27895 & PC10PSNC & Grain-size 10th percentile of particles in sediment [non-carbonate phase] by acidification and particle sizer and analysis of cumulative frequency ordered by phi (coarse to fine) & S0600192 & Grain-size & S0700031 & 10th percentile &  &  &  &  & S026 & in & MAT00129 & sediment [non-carbonate phase] &  &  & S0366 & acidification & S04278 & particle sizer & S050063 & analysis of cumulative frequency ordered by phi (coarse to fine) & MNGS & Sediment grain size parameters & NA & NA\\
\hline
27896 & PC10PSXX & Grain-size 10th percentile of particles in sediment by particle sizer and analysis of cumulative frequency ordered by phi (coarse to fine) & S0600192 & Grain-size & S0700031 & 10th percentile &  &  &  &  & S026 & in & MAT00136 & sediment &  &  &  &  & S04278 & particle sizer & S050063 & analysis of cumulative frequency ordered by phi (coarse to fine) & MNGS & Sediment grain size parameters & NA & NA\\
\hline
27898 & PC16PSXX & Grain-size 16th percentile of particles in sediment by particle sizer and analysis of cumulative frequency ordered by phi (coarse to fine) & S0600192 & Grain-size & S0700032 & 16th percentile &  &  &  &  & S026 & in & MAT00136 & sediment &  &  &  &  & S04278 & particle sizer & S050063 & analysis of cumulative frequency ordered by phi (coarse to fine) & MNGS & Sediment grain size parameters & NA & NA\\
\hline
27899 & PC16SSXX & Grain-size 16th percentile of particles in sediment by sieving and settling tube method and analysis of cumulative frequency ordered by phi (coarse to fine) & S0600192 & Grain-size & S0700032 & 16th percentile &  &  &  &  & S026 & in & MAT00136 & sediment &  &  &  &  & S04321 & sieving and settling tube method & S050063 & analysis of cumulative frequency ordered by phi (coarse to fine) & MNGS & Sediment grain size parameters & NA & NA\\
\hline
27900 & PC25PSXX & Grain-size 25th percentile of particles in sediment by particle sizer and analysis of cumulative frequency ordered by phi (coarse to fine) & S0600192 & Grain-size & S0700033 & 25th percentile &  &  &  &  & S026 & in & MAT00136 & sediment &  &  &  &  & S04278 & particle sizer & S050063 & analysis of cumulative frequency ordered by phi (coarse to fine) & MNGS & Sediment grain size parameters & NA & NA\\
\hline
27901 & PC25SSXX & Grain-size 25th percentile of particles in sediment by sieving and settling tube method and analysis of cumulative frequency ordered by phi (coarse to fine) & S0600192 & Grain-size & S0700033 & 25th percentile &  &  &  &  & S026 & in & MAT00136 & sediment &  &  &  &  & S04321 & sieving and settling tube method & S050063 & analysis of cumulative frequency ordered by phi (coarse to fine) & MNGS & Sediment grain size parameters & NA & NA\\
\hline
27902 & PC50PSNC & Grain-size 50th percentile of particles \{grain-size median\} in sediment [non-carbonate phase] by acidification and particle sizer and analysis of cumulative frequency ordered by phi (coarse to fine) & S0600192 & Grain-size & S0700040 & 50th percentile &  &  &  &  & S026 & in & MAT00129 & sediment [non-carbonate phase] &  &  & S0366 & acidification & S04278 & particle sizer & S050063 & analysis of cumulative frequency ordered by phi (coarse to fine) & MNGS & Sediment grain size parameters & NA & NA\\
\hline
27903 & PC50PSXX & Grain-size 50th percentile of particles \{grain-size median\} in sediment by particle sizer and analysis of cumulative frequency ordered by phi (coarse to fine) & S0600192 & Grain-size & S0700040 & 50th percentile &  &  &  &  & S026 & in & MAT00136 & sediment &  &  &  &  & S04278 & particle sizer & S050063 & analysis of cumulative frequency ordered by phi (coarse to fine) & MNGS & Sediment grain size parameters & NA & NA\\
\hline
27904 & PC50SSXX & Grain-size 50th percentile of particles \{grain-size median\} in sediment by sieving and settling tube method and analysis of cumulative frequency ordered by phi (coarse to fine) & S0600192 & Grain-size & S0700040 & 50th percentile &  &  &  &  & S026 & in & MAT00136 & sediment &  &  &  &  & S04321 & sieving and settling tube method & S050063 & analysis of cumulative frequency ordered by phi (coarse to fine) & MNGS & Sediment grain size parameters & NA & NA\\
\hline
27906 & PC75PSXX & Grain-size 75th percentile of particles in sediment by particle sizer and analysis of cumulative frequency ordered by phi (coarse to fine) & S0600192 & Grain-size & S0700035 & 75th percentile &  &  &  &  & S026 & in & MAT00136 & sediment &  &  &  &  & S04278 & particle sizer & S050063 & analysis of cumulative frequency ordered by phi (coarse to fine) & MNGS & Sediment grain size parameters & NA & NA\\
\hline
27907 & PC75SSXX & Grain-size 75th percentile of particles in sediment by sieving and settling tube method and analysis of cumulative frequency ordered by phi (coarse to fine) & S0600192 & Grain-size & S0700035 & 75th percentile &  &  &  &  & S026 & in & MAT00136 & sediment &  &  &  &  & S04321 & sieving and settling tube method & S050063 & analysis of cumulative frequency ordered by phi (coarse to fine) & MNGS & Sediment grain size parameters & NA & NA\\
\hline
27908 & PC84PSXX & Grain-size 84th percentile of particles in sediment by particle sizer and analysis of cumulative frequency ordered by phi (coarse to fine) & S0600192 & Grain-size & S0700038 & 84th percentile &  &  &  &  & S026 & in & MAT00136 & sediment &  &  &  &  & S04278 & particle sizer & S050063 & analysis of cumulative frequency ordered by phi (coarse to fine) & MNGS & Sediment grain size parameters & NA & NA\\
\hline
27909 & PC84SSXX & Grain-size 84th percentile of particles in sediment by sieving and settling tube method and analysis of cumulative frequency ordered by phi (coarse to fine) & S0600192 & Grain-size & S0700038 & 84th percentile &  &  &  &  & S026 & in & MAT00136 & sediment &  &  &  &  & S04321 & sieving and settling tube method & S050063 & analysis of cumulative frequency ordered by phi (coarse to fine) & MNGS & Sediment grain size parameters & NA & NA\\
\hline
27910 & PC90PSNC & Grain-size 90th percentile of particles in sediment [non-carbonate phase] by acidification and particle sizer and analysis of cumulative frequency ordered by phi (coarse to fine) & S0600192 & Grain-size & S0700037 & 90th percentile &  &  &  &  & S026 & in & MAT00129 & sediment [non-carbonate phase] &  &  & S0366 & acidification & S04278 & particle sizer & S050063 & analysis of cumulative frequency ordered by phi (coarse to fine) & MNGS & Sediment grain size parameters & NA & NA\\
\hline
27911 & PC90PSXX & Grain-size 90th percentile of particles in sediment by particle sizer and analysis of cumulative frequency ordered by phi (coarse to fine) & S0600192 & Grain-size & S0700037 & 90th percentile &  &  &  &  & S026 & in & MAT00136 & sediment &  &  &  &  & S04278 & particle sizer & S050063 & analysis of cumulative frequency ordered by phi (coarse to fine) & MNGS & Sediment grain size parameters & NA & NA\\
\hline
27913 & PC90SSXX & Grain-size 90th percentile of particles in sediment by sieving and settling tube method and analysis of cumulative frequency ordered by phi (coarse to fine) & S0600192 & Grain-size & S0700037 & 90th percentile &  &  &  &  & S026 & in & MAT00136 & sediment &  &  &  &  & S04321 & sieving and settling tube method & S050063 & analysis of cumulative frequency ordered by phi (coarse to fine) & MNGS & Sediment grain size parameters & NA & NA\\
\hline
27915 & PC95PSXX & Grain-size 95th percentile of particles in sediment by particle sizer and analysis of cumulative frequency ordered by phi (coarse to fine) & S0600192 & Grain-size & S0700036 & 95th percentile &  &  &  &  & S026 & in & MAT00136 & sediment &  &  &  &  & S04278 & particle sizer & S050063 & analysis of cumulative frequency ordered by phi (coarse to fine) & MNGS & Sediment grain size parameters & NA & NA\\
\hline
27916 & PC95SSXX & Grain-size 95th percentile of particles in sediment by sieving and settling tube method and analysis of cumulative frequency ordered by phi (coarse to fine) & S0600192 & Grain-size & S0700036 & 95th percentile &  &  &  &  & S026 & in & MAT00136 & sediment &  &  &  &  & S04321 & sieving and settling tube method & S050063 & analysis of cumulative frequency ordered by phi (coarse to fine) & MNGS & Sediment grain size parameters & NA & NA\\
\hline
33147 & SED50VIS & Grain-size median of particles in sediment by visual estimation & S0600192 & Grain-size & S0700004 & median &  &  &  &  & S026 & in & MAT00136 & sediment &  &  &  &  & S04366 & visual estimation &  &  & MNGS & Sediment grain size parameters & NA & NA\\
\hline
33410 & SKGSPSXX & Grain-size skewness of particles in sediment by particle sizer & S0600192 & Grain-size & S0700029 & skewness &  &  &  &  & S026 & in & MAT00136 & sediment &  &  &  &  & S04278 & particle sizer &  &  & MNGS & Sediment grain size parameters & NA & NA\\
\hline
33411 & SKGSSSXX & Grain-size skewness of particles in sediment by sieving and settling tube method & S0600192 & Grain-size & S0700029 & skewness &  &  &  &  & S026 & in & MAT00136 & sediment &  &  &  &  & S04321 & sieving and settling tube method &  &  & MNGS & Sediment grain size parameters & NA & NA\\
\hline
33522 & SND50VIS & Grain-size median of particles (sand size-fraction) in sediment by visual estimation & S0600192 & Grain-size & S0700004 & median &  &  &  &  & S026 & in & MAT00136 & sediment &  &  &  &  & S04366 & visual estimation &  &  & MNGS & Sediment grain size parameters & NA & NA\\
\hline
\end{tabular}

\hypertarget{specials---taxonomical-names}{%
\subsection{Specials - Taxonomical names}\label{specials---taxonomical-names}}

\begin{tabular}[t]{l}
\hline
S25\_preflabel\\
\hline
\\
\hline
Coscinodiscus (ITIS: 2546: WoRMS 148917)\\
\hline
Navicula directa (ITIS: 3669: WoRMS 149467)\\
\hline
Nitzschia (ITIS: 5070: WoRMS 119270)\\
\hline
Rhizosolenia shrubsoleii (ITIS: 2910)\\
\hline
Thalassiosira (ITIS: 2484: WoRMS 148912)\\
\hline
Tropodoneis\\
\hline
Torodinium (ITIS: 10121: WoRMS 109479) [Size: medium]\\
\hline
Sula nebouxii (ITIS: 174702: WoRMS 343959)\\
\hline
Thysanoessa inermis (ITIS: 95573: WoRMS 110708) [Stage: adult Sex: male Subgroup: with spermatophores at the genital aperture or attached to the petasma]\\
\hline
Pseudo-nitzschia (ITIS: 584561: WoRMS 149151)\\
\hline
Umbellula (ITIS: 52384: WoRMS 128499)\\
\hline
Amphidinium (ITIS: 9997: WoRMS 109473)\\
\hline
Ceratium furca (ITIS: 10399: WoRMS 109950)\\
\hline
Ceratium fusus (ITIS: 10400: WoRMS 109951)\\
\hline
Ceratium lineatum (ITIS: 10401: WoRMS 109963)\\
\hline
Gonyaulax polygramma (ITIS: 10371: WoRMS 110035)\\
\hline
Gymnodinium splendens (ITIS: 10037: WoRMS 109832)\\
\hline
Gyrodinium (ITIS: 10077: WoRMS 109476)\\
\hline
Gyrodinium (ITIS: 10077: WoRMS 109476) [Subgroup: sp. B heterotrophic]\\
\hline
Calanus finmarchicus/helgolandicus/glacialis (ITIS: 85263: WoRMS 104152) [Stage: eggs]\\
\hline
Chaetoceros radicans (ITIS: 2822: WoRMS 163112)\\
\hline
Prorocentrum (ITIS: 9877: WoRMS 109566)\\
\hline
Amphorellopsis (WoRMS 136791) [Size: 80-99um]\\
\hline
Mesodinium (ITIS: 46287: WoRMS 179320) [Size: <30um]\\
\hline
Stylocheiron (ITIS: 95556: WoRMS 110678) [Stage: adult Sex: female Subgroup: without spermatophores]\\
\hline
Thalassiosira gravida (ITIS: 2490: WoRMS 149102) [Size: 25um]\\
\hline
Oxytoxum scolopax (ITIS: 10472: WoRMS 110115)\\
\hline
Protoperidinium (ITIS: 10340: WoRMS 109553)\\
\hline
Ptychodiscus noctiluca (ITIS: 331267: WoRMS 109888)\\
\hline
Balanion (WoRMS 292899) [Size: <20um Subgroup: sp. 2]\\
\hline
Torodinium (ITIS: 10121: WoRMS 109479)\\
\hline
Protoperidinium ovum (ITIS: 573488: WoRMS 110243)\\
\hline
Eumicrotremus spinosus (ITIS: 167545: WoRMS 127217)\\
\hline
Amphipoda (ITIS: 93294: WoRMS 1135) [Size: >1000um]\\
\hline
Nematobrachion boopis (ITIS: 95537: WoRMS 110691) [Stage: sub-adult Sex: female]\\
\hline
Euphausia krohnii (ITIS: 660849: WoRMS 110687) [Sex: male]\\
\hline
Urotricha (ITIS: 46243) [Size: <20um]\\
\hline
Chaetoceros atlanticus (ITIS: 2769: WoRMS 149288)\\
\hline
Calanus finmarchicus (ITIS: 85272: WoRMS 104464) [Stage: copepodites C3]\\
\hline
Calanus finmarchicus (ITIS: 85272: WoRMS 104464) [Stage: copepodites C2]\\
\hline
Thysanoessa inermis (ITIS: 95573: WoRMS 110708) [Stage: adult Sex: female]\\
\hline
Hymenaster (ITIS: 157097: WoRMS 123333) [Morphology: visible interbrachial membrane]\\
\hline
Calanus finmarchicus (ITIS: 85272: WoRMS 104464) [Stage: adult Sex: male]\\
\hline
Thysanoessa raschi (ITIS: 95577: WoRMS 416602) [Stage: juvenile]\\
\hline
Ceriantharia (ITIS: 51984: WoRMS 1361) [Size: diameter \textasciitilde{}10cm Morphology: many brown tentacles]\\
\hline
Prorocentrum lenticulatum (WoRMS 232417)\\
\hline
Demospongiae (ITIS: 47528: WoRMS 164811) [Size: height \textasciitilde{}10cm, diameter 30cm Morphology: mass of branches Colour: brown]\\
\hline
Microsetella (ITIS: 86208: WoRMS 115341) [Stage: copepodites plus adults]\\
\hline
Podolampas elegans (ITIS: 10500: WoRMS 110201)\\
\hline
Fragilariopsis kerguelensis (ITIS: 573688: WoRMS 341555) [Subgroup: colonial]\\
\hline
Thalassiosira angustelineata (ITIS: 550490)\\
\hline
Thysanoessa inermis (ITIS: 95573: WoRMS 110708) [Stage: sub-adult+adult Sex: female Subgroup: without spermatophores]\\
\hline
Thysanopoda acutifrons (ITIS: 95583: WoRMS 110712) [Stage: sub-adult+adult Sex: male Subgroup: without external spermatophores]\\
\hline
Mysida (ITIS: 89855: WoRMS 149668) [Stage: adult]\\
\hline
Scomber scombrus (ITIS: 172414: WoRMS 127023) [Stage: eggs]\\
\hline
Trachurus trachurus (ITIS: 168588: WoRMS 126822) [Stage: eggs]\\
\hline
Nematoscelis megalops (ITIS: 95542: WoRMS 110695) [Stage: adult Sex: male Subgroup: without external spermatophores]\\
\hline
Strombidium (ITIS: 46608: WoRMS 101195) [Size: 40-59um Subgroup: sp. 7]\\
\hline
Pseudorca crassidens (ITIS: 180463: WoRMS 137104)\\
\hline
Pseudotriceratium\\
\hline
Stylocheiron maximum (ITIS: 95558: WoRMS 110704) [Sex: female]\\
\hline
Diplopeltopsis (ITIS: 10197: WoRMS 109536) [Stage: cysts]\\
\hline
Lycodes (ITIS: 165255: WoRMS 126104)\\
\hline
aloricate ciliates\\
\hline
Anguilla anguilla (ITIS: 161128: WoRMS 126281)\\
\hline
Pelagostrombidium [Size: 20-39um]\\
\hline
Thysanoessa longicaudata (ITIS: 95575: WoRMS 110709) [Stage: sub-adult+adult Sex: male Subgroup: without external spermatophores]\\
\hline
Gammaproteobacteria (WoRMS 393018) [Subcomponent: nifH gene copies Subgroup: gamma A phylotype]\\
\hline
Ampelisca (ITIS: 93321: WoRMS 101445) [Subgroup: sp. 1]\\
\hline
Amphiuridae (ITIS: 157646: WoRMS 123206) [Stage: juvenile]\\
\hline
Ampharete (ITIS: 67727: WoRMS 129155) [Stage: juvenile]\\
\hline
Astrorhizidae (ITIS: 44046: WoRMS 111958)\\
\hline
Aphroditidae (ITIS: 64359: WoRMS 938) [Stage: juvenile]\\
\hline
Amphictene auricoma (ITIS: 67695: WoRMS 152448)\\
\hline
Ampelisca (ITIS: 93321: WoRMS 101445) [Stage: juvenile]\\
\hline
Aponuphis bilineata (WoRMS 130452)\\
\hline
Abyssorchomene (WoRMS 101585)\\
\hline
Abyssorchomene plebs (WoRMS 237124)\\
\hline
Abyssorchomene rossi (WoRMS 236922)\\
\hline
Abylidae (WoRMS 135336)\\
\hline
Calanus finmarchicus (ITIS: 85272: WoRMS 104464) [Stage: copepodites C4]\\
\hline
Acanthoica quattrospina (ITIS: 2190: WoRMS 235802)\\
\hline
\end{tabular}

\hypertarget{mapping-van-eenheden}{%
\chapter{Mapping van eenheden}\label{mapping-van-eenheden}}

\hypertarget{welke-eenheden-komen-voor}{%
\section{Welke eenheden komen voor?}\label{welke-eenheden-komen-voor}}

Er is uitgegaan van de volgende lijsten met eenheden:

\begin{itemize}
\tightlist
\item
  BODC - De vocabulary \href{https://vocab.nerc.ac.uk/search_nvs/P06/}{P06}
\item
  AQUO - De door AQUO beheerde lijst \href{https://www.aquo.nl/index.php/Id-04f4f467-021b-4218-baa8-9742ed977c61}{Eenheid}
\item
  DDL - Eenheden in de \href{(https://rijkswaterstaat.github.io/wm-ws-dl/\#ophalencatalogus)}{metadata} van in de praktijk uitgeleverde gegevens uit de \href{https://rijkswaterstaat.github.io/wm-ws-dl/\#introduction}{Data Distributielaag}
\end{itemize}

De eenheden in de DDL zouden moeten voldoen aan AQUO, de verwachting is dus dat we alle eenheden in DDL kunnen matchen met eenheden in AQUO.

\hypertarget{vergelijking-ddl-en-aquo}{%
\subsection{Vergelijking DDL en AQUO}\label{vergelijking-ddl-en-aquo}}

De verwachting is dat alle eenheden die in de DDL gebruikt worden in de AQUO Eenheid lijst voorkomen. Dit is onderzocht door de eenheidcodes van de DDL te vergelijken met de eenheidcodes in de AQUO tabel Eenheid. Verreweg de meeste eenheden uit de DDL werden teruggevonden in de AQUO tabel. Tabel \ref{tab:missingDDL} laat de eenheden zien die wel in de DDL gebruikt worden, maar niet teruggevonden konden worden in de AQUO Eenheid tabel, met de bijbehorende parameter.wat.omschrijving uit de DDL.

\begin{table}

\caption{\label{tab:missingDDL}DDL eenheid codes die niet in de AQUO eenheid lijst voorkomen.}
\centering
\begin{tabular}[t]{l|l|l}
\hline
Eenheid.Code & Eenheid.Omschrijving & Parameter\_Wat\_Omschrijving\\
\hline
/l & per liter & Aantal per volume Faecale coliformen in Organisme (biota) t.o.v. natgewicht in Vlees Mytilus edulis/l\\
\hline
RFU & Relative Fluorescence Units & (massa)Concentratie Chlorofyl fluorescentie in rel. fluorescentie eenh.(RFU) in Oppervlaktewater RFU\\
\hline
RFU & Relative Fluorescence Units & Fluorescentie Oppervlaktewater RFU\\
\hline
U & Unit & Fluorescentie Oppervlaktewater U\\
\hline
\end{tabular}
\end{table}

Hierbij moet aangetekend worden dat de eenheid ``/l'' in dit geval ook gelijkgesteld kan worden met de eenheid ``n/l'', die wel in de AQUO lijst voorkomt.

AANBEVELING: verander eenheid ``/l'' in ``n/l'' in de DDL.

\hypertarget{vergelijking-ddl-en-sdn-p0}{%
\subsection{Vergelijking DDL en SDN (P0)}\label{vergelijking-ddl-en-sdn-p0}}

Het ``Eenheid.Code'' veld uit de ddl tabel lijkt op het veld ``altlabel'' uit de SDN tabel. Hierbij zijn alle ``\^{}'' (bijv. in g/m\^{}2) tekens uit de SDN tabel voor de vergelijking verwijderd. Een eerste poging om de twee modellen te mappen wordt daarom gedaan op die twee velden. Het blijkt (tabel \ref{tab:ddl-sdn-eenheden}) dat er 30 codes precies gelijk zijn. De validiteit van deze mapping kan in de tabel \ref{tab:ddl-sdn-eenheden}) geïnspecteerd worden.

\begin{table}

\caption{\label{tab:ddl-sdn-eenheden}Eenheidcodes uit de DDL die direct vergelijkbaar zijn met de eenheidcodes uit de SDN tabel (veldnaam altlabel).}
\centering
\begin{tabular}[t]{l|l|l|l|l}
\hline
Eenheid.Code & Eenheid.Omschrijving & altlabel & preflabel & conceptid\\
\hline
\% & procent & \% & Percent & UPCT\\
\hline
Bq/kg & becquerel per kilogram & Bq/kg & Becquerels per kilogram & UBQK\\
\hline
g & gram & g & Grams & UGRM\\
\hline
g/kg & gram per kilogram & g/kg & Grams per kilogram & UGKG\\
\hline
mg/g & milligram per gram & mg/g & Milligrams per gram & MGPG\\
\hline
mg/kg & milligram per kilogram & mg/kg & Milligrams per kilogram & UMKG\\
\hline
mg/m2 & milligram per vierkante meter & mg/m\textasciicircum{}2 & Milligrams per square metre & UMMS\\
\hline
mm & millimeter & mm & Millimetres & UXMM\\
\hline
ug/kg & microgram per kilogram & ug/kg & Micrograms per kilogram & UUKG\\
\hline
um & micrometer & um & Micrometres (microns) & UMIC\\
\hline
hPa & hectopascal & hPa & Hectopascals & HPAX\\
\hline
m & meter & m & Metres & ULAA\\
\hline
m/s & meter per seconde & m/s & Metres per second & UVAA\\
\hline
cm & centimeter & cm & Centimetres & ULCM\\
\hline
d & dag & d & Days & UTAA\\
\hline
cm2 & vierkante centimeter & cm\textasciicircum{}2 & Square centimetres & SQCM\\
\hline
dB & decibel & dB & Decibels & UDBL\\
\hline
dm & decimeter & dm & Decimetres & ULDM\\
\hline
Hz & hertz & Hz & Hertz & UTHZ\\
\hline
kg/m3 & kilogram per kubieke meter & kg/m\textasciicircum{}3 & Kilograms per cubic metre & UKMC\\
\hline
mBq/l & millibecquerel per liter & mBq/l & Millibecquerels per litre & UMBQ\\
\hline
mg/l & milligram per liter & mg/l & Milligrams per litre & UMGL\\
\hline
min & minuut & min & Minutes & UMIN\\
\hline
mS/m & millisiemens per meter & mS/m & MilliSiemens per metre & MSPM\\
\hline
m3/s & kubieke meter per seconde & m\textasciicircum{}3/s & Cubic metres per second & CMPS\\
\hline
ng/l & nanogram per liter & ng/l & Nanograms per litre & UNGL\\
\hline
NTU & Nephelometric Turbidity Unit & NTU & Nephelometric Turbidity Units & USTU\\
\hline
s & seconde & s & Seconds & UTBB\\
\hline
S/m & siemens per meter & S/m & Siemens per metre & UECA\\
\hline
ug/l & microgram per liter & ug/l & Micrograms per litre & UGPL\\
\hline
h & uur & h & Hours & UHOR\\
\hline
l & liter & l & Litres & ULIT\\
\hline
ng/g & nanogram per gram & ng/g & Nanograms per gram & NGPG\\
\hline
ng/kg & nanogram per kilogram & ng/kg & Nanograms per kilogram & NGKG\\
\hline
\end{tabular}
\end{table}

Eenheden uit de DDL waar met de automatische methode geen exacte match gevonden werd in de SDN eenhedentabel (23) staan in tabel \ref{tab:DDL-SDN-antijoin}.

\begin{table}

\caption{\label{tab:DDL-SDN-antijoin}Eenheidcodes uit de DDL die niet direct vergelijkbaar zijn met de eenheidcodes uit de SDN tabel (veldnaam altlabel).}
\centering
\begin{tabular}[t]{l|l}
\hline
Eenheid.Code & Eenheid.Omschrijving\\
\hline
mg & milligram\\
\hline
n & exemplaren\\
\hline
n/m2 & exemplaren per vierkante meter\\
\hline
n/l & exemplaren per liter\\
\hline
DIMSLS & dimensieloos\\
\hline
graad & graad\\
\hline
oC & graad Celsius\\
\hline
/l & per liter\\
\hline
n/hm & exemplaren per hectometer\\
\hline
FTU & Formazine Turbidity Unit\\
\hline
JTU & Jackson Turbidity Unit\\
\hline
meq/l & milliequivalent per liter\\
\hline
mHz & millihertz\\
\hline
m3/d & kubieke meter per dag\\
\hline
n/ml & exemplaren per milliliter\\
\hline
oD & Duitse graad\\
\hline
RFU & Relative Fluorescence Units\\
\hline
U & Unit\\
\hline
uE & microeinstein\\
\hline
\end{tabular}
\end{table}

\hypertarget{handmatige-mapping}{%
\subsection{Handmatige mapping}\label{handmatige-mapping}}

Voor de niet-gematchte eenheden uit tabel \ref{tab:DDL-SDN-antijoin} wordt hieronder een voorstel gedaan voor een handmatige mapping.

\begin{Shaded}
\begin{Highlighting}[]
\NormalTok{DDL\_SDN\_eenheid\_hand }\OtherTok{\textless{}{-}}\NormalTok{ DDL\_SDN\_verschil }\SpecialCharTok{\%\textgreater{}\%}
  \FunctionTok{mutate}\NormalTok{(}\AttributeTok{altlabel =} 
           \FunctionTok{case\_when}\NormalTok{(}
\NormalTok{             Eenheid.Code }\SpecialCharTok{==} \StringTok{"mg"} \SpecialCharTok{\textasciitilde{}} \ConstantTok{NA\_character\_}\NormalTok{, }\CommentTok{\# Grams (UGRM) en Micrograms (UGUG) komt wel voor}
\NormalTok{             Eenheid.Code }\SpecialCharTok{==} \StringTok{"n"} \SpecialCharTok{\textasciitilde{}} \ConstantTok{NA\_character\_}\NormalTok{, }\CommentTok{\# in BODC altijd gerelateerd aan volume of lengte}
\NormalTok{             Eenheid.Code }\SpecialCharTok{==} \StringTok{"n/m2"} \SpecialCharTok{\textasciitilde{}} \StringTok{"\#/m\^{}2"}\NormalTok{,}
\NormalTok{             Eenheid.Code }\SpecialCharTok{==} \StringTok{"n/l"} \SpecialCharTok{\textasciitilde{}} \StringTok{"\#/l"}\NormalTok{,}
\NormalTok{             Eenheid.Code }\SpecialCharTok{==} \StringTok{"DIMSLS"} \SpecialCharTok{\textasciitilde{}} \StringTok{""}\NormalTok{,}
\NormalTok{             Eenheid.Code }\SpecialCharTok{==} \StringTok{"graad"} \SpecialCharTok{\textasciitilde{}} \StringTok{"deg"}\NormalTok{,}
\NormalTok{             Eenheid.Code }\SpecialCharTok{==} \StringTok{"oC"} \SpecialCharTok{\textasciitilde{}} \StringTok{"degC"}\NormalTok{,}
\NormalTok{             Eenheid.Code }\SpecialCharTok{==} \StringTok{"/l"} \SpecialCharTok{\textasciitilde{}} \StringTok{"\#/l"}\NormalTok{,}
\NormalTok{             Eenheid.Code }\SpecialCharTok{==} \StringTok{"n/hm"} \SpecialCharTok{\textasciitilde{}} \ConstantTok{NA\_character\_}\NormalTok{, }\CommentTok{\# (\#/m komt wel voor)}
\NormalTok{             Eenheid.Code }\SpecialCharTok{==} \StringTok{"FTU"} \SpecialCharTok{\textasciitilde{}} \ConstantTok{NA\_character\_}\NormalTok{, }\CommentTok{\# komt ook niet in AQUO voor}
\NormalTok{             Eenheid.Code }\SpecialCharTok{==} \StringTok{"JTU"} \SpecialCharTok{\textasciitilde{}} \ConstantTok{NA\_character\_}\NormalTok{, }\CommentTok{\# komt ook niet in AQUO voor}
\NormalTok{             Eenheid.Code }\SpecialCharTok{==} \StringTok{"meq/l"} \SpecialCharTok{\textasciitilde{}} \StringTok{"mEquiv/l"}\NormalTok{,}
\NormalTok{             Eenheid.Code }\SpecialCharTok{==} \StringTok{"mHz"} \SpecialCharTok{\textasciitilde{}} \ConstantTok{NA\_character\_}\NormalTok{, }\CommentTok{\# (Hz komt wel voor)}
\NormalTok{             Eenheid.Code }\SpecialCharTok{==} \StringTok{"m3/d"} \SpecialCharTok{\textasciitilde{}} \ConstantTok{NA\_character\_}\NormalTok{,  }\CommentTok{\# (m\^{}3/s komt wel voor)}
\NormalTok{             Eenheid.Code }\SpecialCharTok{==} \StringTok{"n/ml"} \SpecialCharTok{\textasciitilde{}} \StringTok{"\#/ml"}\NormalTok{,}
\NormalTok{             Eenheid.Code }\SpecialCharTok{==} \StringTok{"oD"} \SpecialCharTok{\textasciitilde{}} \ConstantTok{NA\_character\_}\NormalTok{,  }\CommentTok{\# (duitse graad, komt in mariene metingen niet voor)}
\NormalTok{             Eenheid.Code }\SpecialCharTok{==} \StringTok{"RFU"} \SpecialCharTok{\textasciitilde{}} \ConstantTok{NA\_character\_}\NormalTok{, }\CommentTok{\# komt ook niet in AQUO voor}
\NormalTok{             Eenheid.Code }\SpecialCharTok{==} \StringTok{"U"} \SpecialCharTok{\textasciitilde{}} \ConstantTok{NA\_character\_}\NormalTok{,   }\CommentTok{\# komt ook niet in AQUO voor}
\NormalTok{             Eenheid.Code }\SpecialCharTok{==} \StringTok{"uE"}  \SpecialCharTok{\textasciitilde{}} \ConstantTok{NA\_character\_}  \CommentTok{\# waarschijnlijk uE/cm\^{}2/s, vraag is of het voorkomt in marien milieu}
\NormalTok{           )}
\NormalTok{         )}
\end{Highlighting}
\end{Shaded}

Er is een probleem bij het mappen van de eenheid ``DIMSLS'' die in de DDL gebruikt wordt. Weliswaar is er een eenheid met preflabel = ``Dimensionless'' in de P06 tabel, maar dit leidt vermoedelijk niet tot een goede mapping.

De eenheid ``DIMSLS'' wordt in de DDL gebruikt voor de parameter.wat.omschrijvingen zoals in tabel \ref{tab:DDL-DIMSLS}). Hier zitten grootheden als Extinctie, Zuurgraad en Saliniteit tussen, die in SDN termen een eigen eenheid hebben. In de P06 tabel zijn de eenheden ``pH units'', ``per metre'' en ``g/kg'' te vinden die gangbaar zijn voor respectievelijk Zuurgraad, Extinctie en Saliniteit. Een mapping hiervan kan in de huidige situatie niet alleen op grond van eenheidcode, maar moet ook de grootheid meenemen.

Afgezien van het feit dat er geen een-op-een mapping mogelijk is, is ook de meer inhoudelijke vraag belangrijk of ``dimensieloos'' wel informatief is in deze gevallen.

\begin{table}

\caption{\label{tab:DDL-DIMSLS}DDL Grootheid.Omschrijving waarden die de eenheid DIMSLS hebben. }
\centering
\begin{tabular}[t]{l|l}
\hline
Parameter\_Wat\_Omschrijving & Grootheid.Omschrijving\\
\hline
Bewolkingsgraad Lucht & Waarde is niet van toepassing\\
\hline
Intersexindex Organisme (biota) Littorina littorea & Intersexindex\\
\hline
Vas Deferens Sequence Index (zaadleider index) Organisme (biota) Nassarius reticulatus & Vas Deferens Sequence Index (zaadleider index)\\
\hline
Aanwezigheid Geur in Oppervlaktewater & Aanwezigheid\\
\hline
Aanwezigheid Kleur in Oppervlaktewater & Aanwezigheid\\
\hline
Aanwezigheid Olie in Oppervlaktewater & Aanwezigheid\\
\hline
Aanwezigheid Schuim in Oppervlaktewater & Aanwezigheid\\
\hline
Aanwezigheid Vuil in Oppervlaktewater & Aanwezigheid\\
\hline
Aantal golven in 20 minuten periode Oppervlaktewater & Aantal golven in 20 minuten periode\\
\hline
Anomalie Oppervlaktewater Uitgedrukt in Gadolinium na filtratie in & Anomalie\\
\hline
Aantal vrijheidsgraden bij het variantiedichtheidspectrum Oppervlaktewater & Aantal vrijheidsgraden bij het variantiedichtheidspectrum\\
\hline
Aantal vrijheidsgraden behorend bij het golfrichtingspectrum Oppervlaktewater & Aantal vrijheidsgraden behorend bij het golfrichtingspectrum\\
\hline
Extinctie Oppervlaktewater & Extinctie\\
\hline
Geurverdunningsfactor Oppervlaktewater & Geurverdunningsfactor\\
\hline
Getijextreemtype Oppervlaktewater & Waarde is niet van toepassing\\
\hline
Kleur intensiteit Oppervlaktewater uitgedrukt volgens PtCo-schaal in & Waarde is niet van toepassing\\
\hline
Zuurgraad Oppervlaktewater & Zuurgraad\\
\hline
Reuk Oppervlaktewater & Reuk\\
\hline
Saliniteit Oppervlaktewater & Saliniteit\\
\hline
\end{tabular}
\end{table}

\textbf{Aanbeveling}

De AQUO eenheidcode DIMSLS (Dimensieloos) die in de DDL gebruikt wordt is misschien formeel correct, maar is niet informatief. Bovendien zorgt dit ervoor dat eenheden niet te converteren zijn naar BODC termen. Het verdient daarom aanbeveling om deze te veranderen naar een meer informatieve eenheid. Voor een koppeling naar BODC (beperkt tot waarnemingen in mariene milie) zijn vooral eenheden voor de volgende grootheden van belang:

\begin{longtable}[]{@{}lll@{}}
\toprule
Grootheid & Huidige eenheid DDL & Voorstel nieuwe eenheid \\
\midrule
\endhead
Saliniteit & DIMSLS & g/kg \\
Zuurgraad & DIMSLS & pH-eenheid \\
Extinctie & DIMSLS & /m \\
\bottomrule
\end{longtable}

\hypertarget{uiteindelijke-mapping-van-eenheden.}{%
\subsection{Uiteindelijke mapping van eenheden.}\label{uiteindelijke-mapping-van-eenheden.}}

Door de tabellen met de automatische en de handmatige mapping samen te voegen, wordt een mappingtabel gemaakt die zo veel mogelijk compleet is.

\begin{table}

\caption{\label{tab:unnamed-chunk-3}Samengevoegde tabel (automatisch en handmatig gematcht. }
\centering
\begin{tabular}[t]{l|l|l|l|l}
\hline
Eenheid.Code & Eenheid.Omschrijving & altlabel & preflabel & uri\\
\hline
\% & procent & \% & Percent & http://vocab.nerc.ac.uk/collection/P06/current/UPCT/\\
\hline
Bq/kg & becquerel per kilogram & Bq/kg & Becquerels per kilogram & http://vocab.nerc.ac.uk/collection/P06/current/UBQK/\\
\hline
g & gram & g & Grams & http://vocab.nerc.ac.uk/collection/P06/current/UGRM/\\
\hline
g/kg & gram per kilogram & g/kg & Grams per kilogram & http://vocab.nerc.ac.uk/collection/P06/current/UGKG/\\
\hline
mg/g & milligram per gram & mg/g & Milligrams per gram & http://vocab.nerc.ac.uk/collection/P06/current/MGPG/\\
\hline
mg/kg & milligram per kilogram & mg/kg & Milligrams per kilogram & http://vocab.nerc.ac.uk/collection/P06/current/UMKG/\\
\hline
mg/m2 & milligram per vierkante meter & mg/m\textasciicircum{}2 & Milligrams per square metre & http://vocab.nerc.ac.uk/collection/P06/current/UMMS/\\
\hline
mm & millimeter & mm & Millimetres & http://vocab.nerc.ac.uk/collection/P06/current/UXMM/\\
\hline
ug/kg & microgram per kilogram & ug/kg & Micrograms per kilogram & http://vocab.nerc.ac.uk/collection/P06/current/UUKG/\\
\hline
um & micrometer & um & Micrometres (microns) & http://vocab.nerc.ac.uk/collection/P06/current/UMIC/\\
\hline
hPa & hectopascal & hPa & Hectopascals & http://vocab.nerc.ac.uk/collection/P06/current/HPAX/\\
\hline
m & meter & m & Metres & http://vocab.nerc.ac.uk/collection/P06/current/ULAA/\\
\hline
m/s & meter per seconde & m/s & Metres per second & http://vocab.nerc.ac.uk/collection/P06/current/UVAA/\\
\hline
cm & centimeter & cm & Centimetres & http://vocab.nerc.ac.uk/collection/P06/current/ULCM/\\
\hline
d & dag & d & Days & http://vocab.nerc.ac.uk/collection/P06/current/UTAA/\\
\hline
cm2 & vierkante centimeter & cm\textasciicircum{}2 & Square centimetres & http://vocab.nerc.ac.uk/collection/P06/current/SQCM/\\
\hline
dB & decibel & dB & Decibels & http://vocab.nerc.ac.uk/collection/P06/current/UDBL/\\
\hline
dm & decimeter & dm & Decimetres & http://vocab.nerc.ac.uk/collection/P06/current/ULDM/\\
\hline
Hz & hertz & Hz & Hertz & http://vocab.nerc.ac.uk/collection/P06/current/UTHZ/\\
\hline
kg/m3 & kilogram per kubieke meter & kg/m\textasciicircum{}3 & Kilograms per cubic metre & http://vocab.nerc.ac.uk/collection/P06/current/UKMC/\\
\hline
mBq/l & millibecquerel per liter & mBq/l & Millibecquerels per litre & http://vocab.nerc.ac.uk/collection/P06/current/UMBQ/\\
\hline
mg/l & milligram per liter & mg/l & Milligrams per litre & http://vocab.nerc.ac.uk/collection/P06/current/UMGL/\\
\hline
min & minuut & min & Minutes & http://vocab.nerc.ac.uk/collection/P06/current/UMIN/\\
\hline
mS/m & millisiemens per meter & mS/m & MilliSiemens per metre & http://vocab.nerc.ac.uk/collection/P06/current/MSPM/\\
\hline
m3/s & kubieke meter per seconde & m\textasciicircum{}3/s & Cubic metres per second & http://vocab.nerc.ac.uk/collection/P06/current/CMPS/\\
\hline
ng/l & nanogram per liter & ng/l & Nanograms per litre & http://vocab.nerc.ac.uk/collection/P06/current/UNGL/\\
\hline
NTU & Nephelometric Turbidity Unit & NTU & Nephelometric Turbidity Units & http://vocab.nerc.ac.uk/collection/P06/current/USTU/\\
\hline
s & seconde & s & Seconds & http://vocab.nerc.ac.uk/collection/P06/current/UTBB/\\
\hline
S/m & siemens per meter & S/m & Siemens per metre & http://vocab.nerc.ac.uk/collection/P06/current/UECA/\\
\hline
ug/l & microgram per liter & ug/l & Micrograms per litre & http://vocab.nerc.ac.uk/collection/P06/current/UGPL/\\
\hline
h & uur & h & Hours & http://vocab.nerc.ac.uk/collection/P06/current/UHOR/\\
\hline
l & liter & l & Litres & http://vocab.nerc.ac.uk/collection/P06/current/ULIT/\\
\hline
ng/g & nanogram per gram & ng/g & Nanograms per gram & http://vocab.nerc.ac.uk/collection/P06/current/NGPG/\\
\hline
ng/kg & nanogram per kilogram & ng/kg & Nanograms per kilogram & http://vocab.nerc.ac.uk/collection/P06/current/NGKG/\\
\hline
mg & milligram & NA & NA & http://vocab.nerc.ac.uk/collection/P06/current/NA/\\
\hline
n & exemplaren & NA & NA & http://vocab.nerc.ac.uk/collection/P06/current/NA/\\
\hline
n/m2 & exemplaren per vierkante meter & \#/m\textasciicircum{}2 & NA & http://vocab.nerc.ac.uk/collection/P06/current/NA/\\
\hline
n/l & exemplaren per liter & \#/l & NA & http://vocab.nerc.ac.uk/collection/P06/current/NA/\\
\hline
DIMSLS & dimensieloos &  & NA & http://vocab.nerc.ac.uk/collection/P06/current/NA/\\
\hline
graad & graad & deg & NA & http://vocab.nerc.ac.uk/collection/P06/current/NA/\\
\hline
oC & graad Celsius & degC & NA & http://vocab.nerc.ac.uk/collection/P06/current/NA/\\
\hline
/l & per liter & \#/l & NA & http://vocab.nerc.ac.uk/collection/P06/current/NA/\\
\hline
n/hm & exemplaren per hectometer & NA & NA & http://vocab.nerc.ac.uk/collection/P06/current/NA/\\
\hline
FTU & Formazine Turbidity Unit & NA & NA & http://vocab.nerc.ac.uk/collection/P06/current/NA/\\
\hline
JTU & Jackson Turbidity Unit & NA & NA & http://vocab.nerc.ac.uk/collection/P06/current/NA/\\
\hline
meq/l & milliequivalent per liter & mEquiv/l & NA & http://vocab.nerc.ac.uk/collection/P06/current/NA/\\
\hline
mHz & millihertz & NA & NA & http://vocab.nerc.ac.uk/collection/P06/current/NA/\\
\hline
m3/d & kubieke meter per dag & NA & NA & http://vocab.nerc.ac.uk/collection/P06/current/NA/\\
\hline
n/ml & exemplaren per milliliter & \#/ml & NA & http://vocab.nerc.ac.uk/collection/P06/current/NA/\\
\hline
oD & Duitse graad & NA & NA & http://vocab.nerc.ac.uk/collection/P06/current/NA/\\
\hline
RFU & Relative Fluorescence Units & NA & NA & http://vocab.nerc.ac.uk/collection/P06/current/NA/\\
\hline
U & Unit & NA & NA & http://vocab.nerc.ac.uk/collection/P06/current/NA/\\
\hline
uE & microeinstein & NA & NA & http://vocab.nerc.ac.uk/collection/P06/current/NA/\\
\hline
\end{tabular}
\end{table}

\hypertarget{kwaliteitsoordeel}{%
\chapter{Kwaliteitsoordeel}\label{kwaliteitsoordeel}}

In AQUO kunnen verschillende kwaliteitsoordelen worden toegekend aan een meetwaarde (tabel \ref{tab:aquoKwaliteitsoordeel}). Het is niet eenvoudig om te bepalen welke codes in de praktijk worden toegepast. Het gebruik in de Data Distributielaag kan alleen bepaald worden wanneer data daadwerkelijk gedownloadt worden.

\begin{table}

\caption{\label{tab:aquoKwaliteitsoordeel}Aquo kwaliteitsoordeelcodes en -omschrijvingen.}
\centering
\begin{tabular}[t]{l|l|l}
\hline
  & Codes & Omschrijving\\
\hline
1 & 00 & Normale waarde\\
\hline
5 & 03 & Waarde heeft een grotere spreiding dan beschreven\\
\hline
7 & 04 & Bepaald met hele detectiegrens\\
\hline
8 & 05 & Bepaald met halve detectiegrens\\
\hline
12 & 06 & Bepaald met nul waarde voor detectiegrens\\
\hline
14 & 07 & Waarde is verhoogde rapportagegrens\\
\hline
2 & 10 & In de ruimte geïnterpoleerde waarde\\
\hline
3 & 20 & In de tijd geïnterpoleerde waarde\\
\hline
4 & 25 & In ruimte en tijd geïnterpoleerde waarde\\
\hline
6 & 30 & Waarde beïnvloedt door ruimtelijke activiteiten\\
\hline
9 & 50 & Niet-plausibele waarde\\
\hline
10 & 55 & Gevlagde waarde, bepaald met halve detectiegrens\\
\hline
11 & 56 & Gevlagde waarde, bepaald met nul waarde voor detectiegrens\\
\hline
13 & 61 & Gecorrigeerde waarde op basis van systematische fout\\
\hline
15 & 70 & Afgekeurde waarde op basis van trendgedrag\\
\hline
16 & 71 & Afgekeurde waarde op basis van harde grenzen\\
\hline
17 & 72 & Afgekeurde waarde op basis van ionenbalans\\
\hline
18 & 73 & Afgekeurde waarde op basis van springerigheid\\
\hline
19 & 74 & Afgekeurde waarde op basis van levendigheid\\
\hline
20 & 75 & Afgekeurde waarde op basis van uitbijter\\
\hline
21 & 76 & Afgekeurde waarde o.b.v. correlatie tussen meetlocaties\\
\hline
22 & 77 & Afgekeurde waarde op basis van correlatie tussen parameters\\
\hline
23 & 78 & Afgekeurde waarde opgeloste parameter is hoger dan totaal\\
\hline
24 & 79 & Afgekeurde waarde agv rapportage lager dan rapportagegrens\\
\hline
25 & 80 & Afgekeurde waarde op basis van trendbreuk\\
\hline
26 & 81 & Afgekeurde waarde op basis van controlemeting\\
\hline
27 & 82 & Afgekeurde waarde op basis van waterbalans\\
\hline
28 & 83 & Afwijkende waarde als gevolg van buiten meetbereik\\
\hline
29 & 84 & Afwijkende waarde als gevolg van foutieve nulpunt\\
\hline
30 & 90 & Afwijkende waarde na validatie goedgekeurd\\
\hline
31 & 91 & Afwijkende waarde i.v.m. extreme situatie (calamiteit)\\
\hline
32 & 98 & Waarde bepaald op onvolledige basis\\
\hline
33 & 99 & Hiaat waarde\\
\hline
\end{tabular}
\end{table}

De corresponderende tabel met algemene kwalititeitsoordelen van BODC is de tabel \href{http://vocab.nerc.ac.uk/collection/L20/current/}{L20} (figuur \ref{tab:bodcKwaliteitsoordeel}). BODC hanteert hiernaast tabellen met meer specifieke ``quality flags'' voor bepaalde apparaten of organisaties (IODE). Hier gaan we nu niet verder op in.

\begin{tabular}[t]{l|l|l}
\hline
conceptid & preflabel & definition\\
\hline
0 & no quality control & No quality control procedures have been applied to the data value. This is the initial status for all data values entering the working archive.\\
\hline
1 & good value & Good quality data value that is not part of any identified malfunction and has been verified as consistent with real phenomena during the quality control process.\\
\hline
2 & probably good value & Data value that is probably consistent with real phenomena but this is unconfirmed or data value forming part of a malfunction that is considered too small to affect the overall quality of the data object of which it is a part.\\
\hline
3 & probably bad value & Data value recognised as unusual during quality control that forms part of a feature that is probably inconsistent with real phenomena.\\
\hline
4 & bad value & An obviously erroneous data value.\\
\hline
5 & changed value & Data value adjusted during quality control.  Best practice strongly recommends that the value before the change be preserved in the data or its accompanying metadata.\\
\hline
6 & value below detection & The level of the measured phenomenon was less than the limit of detection (LoD) for the method employed to measure it. The accompanying value is the detection limit for the technique or zero if that value is unknown.\\
\hline
7 & value in excess & The level of the measured phenomenon was too large to be quantified by the technique employed to measure it.  The accompanying value is the measurement limit for the technique.\\
\hline
8 & interpolated value & This value has been derived by interpolation from other values in the data object.\\
\hline
9 & missing value & The data value is missing. There should be no accompanying value in ODV format files. The accompanying value in SeaDataNet NetCDF data must be the absent data representation specified by the \_FillValue parameter attribute and lie outside the range of data not flagged bad (4) or probably bad (3).\\
\hline
A & value phenomenon uncertain & There is uncertainty in the description of the measured phenomenon associated with the value such as chemical species or biological entity.\\
\hline
Q & value below limit of quantification & The level of the measured phenomenon was less than the limit of quantification (LoQ). The accompanying value is the limit of quantification for the analytical method.\\
\hline
\end{tabular}

In eerder werk (NWDM) is een mapping gemaakt op basis van data uit de DDL voor Noordzee en Waddenzee (fysische-chemische data). Het voorstel is om deze als basis te gebruiken. De tabel

Dit

\begin{table}

\caption{\label{tab:Kwaliteitscodecomparison}Voorgestelde apping van kwaliteitscodes.}
\centering
\begin{tabular}[t]{l|l|l|l}
\hline
aquo\_code & aquo\_omschrijving & conceptid & preflabel\\
\hline
00 & Normale waarde & 1 & good value\\
\hline
03 & Waarde heeft een grotere spreiding dan beschreven & 2 & probably good value\\
\hline
04 & Bepaald met hele detectiegrens & 2 & probably good value\\
\hline
05 & Bepaald met halve detectiegrens & 2 & probably good value\\
\hline
06 & Bepaald met nul waarde voor detectiegrens & 2 & probably good value\\
\hline
07 & Waarde is verhoogde rapportagegrens & 2 & probably good value\\
\hline
10 & In de ruimte geïnterpoleerde waarde & 2 & probably good value\\
\hline
20 & In de tijd geïnterpoleerde waarde & 2 & probably good value\\
\hline
25 & In ruimte en tijd geïnterpoleerde waarde & 2 & probably good value\\
\hline
30 & Waarde beïnvloedt door ruimtelijke activiteiten & 2 & probably good value\\
\hline
50 & Niet-plausibele waarde & 3 & probably bad value\\
\hline
98 & Waarde bepaald op onvolledige basis & 3 & probably bad value\\
\hline
99 & Hiaat waarde & 4 & bad value\\
\hline
55 & Gevlagde waarde, bepaald met halve detectiegrens & NA & NA\\
\hline
56 & Gevlagde waarde, bepaald met nul waarde voor detectiegrens & NA & NA\\
\hline
61 & Gecorrigeerde waarde op basis van systematische fout & NA & NA\\
\hline
70 & Afgekeurde waarde op basis van trendgedrag & NA & NA\\
\hline
71 & Afgekeurde waarde op basis van harde grenzen & NA & NA\\
\hline
72 & Afgekeurde waarde op basis van ionenbalans & NA & NA\\
\hline
73 & Afgekeurde waarde op basis van springerigheid & NA & NA\\
\hline
74 & Afgekeurde waarde op basis van levendigheid & NA & NA\\
\hline
75 & Afgekeurde waarde op basis van uitbijter & NA & NA\\
\hline
76 & Afgekeurde waarde o.b.v. correlatie tussen meetlocaties & NA & NA\\
\hline
77 & Afgekeurde waarde op basis van correlatie tussen parameters & NA & NA\\
\hline
78 & Afgekeurde waarde opgeloste parameter is hoger dan totaal & NA & NA\\
\hline
79 & Afgekeurde waarde agv rapportage lager dan rapportagegrens & NA & NA\\
\hline
80 & Afgekeurde waarde op basis van trendbreuk & NA & NA\\
\hline
81 & Afgekeurde waarde op basis van controlemeting & NA & NA\\
\hline
82 & Afgekeurde waarde op basis van waterbalans & NA & NA\\
\hline
83 & Afwijkende waarde als gevolg van buiten meetbereik & NA & NA\\
\hline
84 & Afwijkende waarde als gevolg van foutieve nulpunt & NA & NA\\
\hline
90 & Afwijkende waarde na validatie goedgekeurd & NA & NA\\
\hline
91 & Afwijkende waarde i.v.m. extreme situatie (calamiteit) & NA & NA\\
\hline
NA & NA & 0 & no quality control\\
\hline
NA & NA & 5 & changed value\\
\hline
NA & NA & 6 & value below detection\\
\hline
NA & NA & 7 & value in excess\\
\hline
NA & NA & 8 & interpolated value\\
\hline
NA & NA & 9 & missing value\\
\hline
NA & NA & A & value phenomenon uncertain\\
\hline
NA & NA & Q & value below limit of quantification\\
\hline
\end{tabular}
\end{table}

\hypertarget{bemonstering}{%
\chapter{Bemonstering}\label{bemonstering}}

\hypertarget{bemonsteringsapparaat-of-veldapparaat}{%
\section{Bemonsteringsapparaat of Veldapparaat}\label{bemonsteringsapparaat-of-veldapparaat}}

\begin{verbatim}
#> [1] 0
\end{verbatim}

\begin{table}

\caption{\label{tab:unnamed-chunk-2}Lijst met bemonsteringsapparaten in AQUO domeintabel "bemonsterinsapparaat"}
\centering
\begin{tabular}[t]{l|l}
\hline
  & Omschrijving\\
\hline
1 & Edelmanboor\\
\hline
21 & Riversideboor\\
\hline
64 & Puinboor\\
\hline
2 & Ramguts\\
\hline
13 & Gutsboor\\
\hline
14 & Steektoestel zonder folie\\
\hline
15 & Aqualock\\
\hline
16 & Avegaarboor\\
\hline
17 & Beeker-sampler\\
\hline
18 & Begeman-sampler\\
\hline
19 & Box-corer\\
\hline
20 & Continuous soil sampler\\
\hline
22 & Drukkend boorsysteem\\
\hline
23 & Eckman-Birge happer\\
\hline
24 & Foliesteektoestel\\
\hline
25 & Grindboor\\
\hline
26 & Hamerend boorsysteem\\
\hline
27 & Handpuls\\
\hline
28 & Holle avegaar\\
\hline
29 & Jenkins-mudsampler\\
\hline
30 & Kernboor\\
\hline
31 & Mostap\\
\hline
32 & Multi-sampler\\
\hline
33 & Akkermanboor\\
\hline
34 & Pulsboor\\
\hline
35 & Sonisch boorsysteem\\
\hline
36 & Spitsmuisboor\\
\hline
37 & Valbom\\
\hline
38 & Van der Horst-steektoestel\\
\hline
39 & Van Veen happer\\
\hline
40 & Veenboor\\
\hline
41 & Vibro-corer\\
\hline
42 & Vrijwitboor\\
\hline
43 & Zuigerboor\\
\hline
44 & Akoestische zandtransportmeter\\
\hline
45 & Automatisch monstername apparaat\\
\hline
46 & Bodemschaaf\\
\hline
47 & Boomkor\\
\hline
48 & Dompelpomp\\
\hline
49 & Doorstroomcentrifuge\\
\hline
50 & Elektrisch schepnet\\
\hline
51 & Emmer\\
\hline
52 & Flushing sampler\\
\hline
53 & Fuik\\
\hline
54 & Handnet\\
\hline
55 & Korf\\
\hline
56 & Kuil\\
\hline
57 & Modderpuls\\
\hline
58 & Mosselkorf\\
\hline
59 & Pelagic trawl\\
\hline
60 & Piston-corer\\
\hline
61 & Planktonnet\\
\hline
62 & Poliepgrijper\\
\hline
63 & Pollepel\\
\hline
65 & Pomp\\
\hline
66 & Roset sampler\\
\hline
67 & Schietfuik\\
\hline
68 & Sedimentval\\
\hline
69 & Siliconen rubber passive sampler\\
\hline
70 & Slib sampler\\
\hline
71 & Snoeischaar\\
\hline
72 & Sondeerapparaat\\
\hline
73 & Speeddisk passive sampler\\
\hline
74 & Spiraal\\
\hline
75 & Steekbuis\\
\hline
76 & Vacuümpomp\\
\hline
77 & Cilindrisch monsternemingstoestel\\
\hline
78 & Werpkorf\\
\hline
79 & Elektrisch schepnet met keernetten\\
\hline
80 & Stortkuil\\
\hline
81 & Wonderkuil\\
\hline
82 & Zegen\\
\hline
83 & Zegen met keernetten\\
\hline
84 & Elektrostramienkor\\
\hline
85 & Zuigkor\\
\hline
86 & Hamon happer\\
\hline
87 & Videocamera\\
\hline
88 & Atoomkuil\\
\hline
89 & Ankerkuil\\
\hline
90 & Graafmachine\\
\hline
3 & Schep\\
\hline
4 & Wortelboor\\
\hline
5 & Veenguts\\
\hline
6 & Geoprobe\\
\hline
7 & Maatbeker\\
\hline
8 & Peristaltische pomp\\
\hline
9 & Telescopische stok met flessenhouder\\
\hline
10 & Stootijzer\\
\hline
11 & Vacuüm steekbuis\\
\hline
12 & Macrozoöbenthos zuiger\\
\hline
\end{tabular}
\end{table}

BODC onderhoudt verschillende tabellen waar bemonsteringsapparaten in voorkomen. De meest gedetailleerde tabel is de \href{http://vocab.nerc.ac.uk/collection/L22/current/}{SeaVoX Device Catalogue L22}. Een minder gedetailleerde (broader) tabel is \href{http://vocab.nerc.ac.uk/collection/L05/current/}{SeaDataNet device category}.

De mapping die hier gepresenteerd wordt is slechts

\begin{table}

\caption{\label{tab:bodcBemonsteringsapparaat}Eerste elementen van BODC L22 tabel SeaVoX Device Catalogue.}
\centering
\begin{tabular}[t]{l|l}
\hline
conceptid & preflabel\\
\hline
NETT0001 & Adriatic plankton sampler - Krsinic (1990)\\
\hline
NETT0002 & High-speed sampler - Apstein (1906)\\
\hline
NETT0003 & Apstein net as described by  Apstein (1896); Dakin (1908)\\
\hline
NETT0004 & Autosampling and Recording Instrumental Environmental Sampler - Dunn et al. (1993)\\
\hline
NETT0005 & Automatic high-speed plankton sampler - Williamson (1962, 1963)\\
\hline
NETT0006 & Closing net - Barnes (1953)\\
\hline
\end{tabular}
\end{table}

In eerdere transformaties van Nederlandse data naar EMODnet Biologie zijn wel eens mappings gemaakt. Hieruit is de ad hoc handmatige mapping hieronder samengesteld.

\begin{tabular}[t]{l|l|l}
\hline
aquo\_omschrijving & aquo\_cijfercode & bodc\_code\\
\hline
Pelagic trawl & 62 & http://vocab.nerc.ac.uk/collection/L05/current/23/\\
\hline
Videocamera & 96 & http://vocab.nerc.ac.uk/collection/L05/current/311/\\
\hline
Vacuüm steekbuis & 108 & http://vocab.nerc.ac.uk/collection/L05/current/391/\\
\hline
Piston-corer & 13 & http://vocab.nerc.ac.uk/collection/L05/current/51/\\
\hline
Fuik & 57 & http://vocab.nerc.ac.uk/collection/L05/current/63/\\
\hline
Zuigkor & 94 & http://vocab.nerc.ac.uk/collection/L05/current/64/\\
\hline
Pomp & 68 & http://vocab.nerc.ac.uk/collection/L22/current/NETT0173/\\
\hline
Bodemschaaf & 45 & http://vocab.nerc.ac.uk/collection/L22/current/NETT0190/\\
\hline
Emmer & 33 & http://vocab.nerc.ac.uk/collection/L22/current/TOOL0536/\\
\hline
Boomkor & 55 & http://vocab.nerc.ac.uk/collection/L22/current/TOOL0651/\\
\hline
Van Veen happer & 15 & http://vocab.nerc.ac.uk/collection/L22/current/TOOL0653/\\
\hline
Hamon happer & 95 & http://vocab.nerc.ac.uk/collection/L22/current/TOOL0960/\\
\hline
Box-corer & 46 & http://vocab.nerc.ac.uk/collection/L22/current/TOOL1177/\\
\hline
Edelmanboor & 20 & NA\\
\hline
Riversideboor & 25 & NA\\
\hline
Puinboor & 87 & NA\\
\hline
Ramguts & 27 & NA\\
\hline
Gutsboor & 31 & NA\\
\hline
Steektoestel zonder folie & 53 & NA\\
\hline
Aqualock & 83 & NA\\
\hline
Avegaarboor & 54 & NA\\
\hline
Beeker-sampler & 12 & NA\\
\hline
Begeman-sampler & 44 & NA\\
\hline
Continuous soil sampler & 72 & NA\\
\hline
Drukkend boorsysteem & 73 & NA\\
\hline
Eckman-Birge happer & 41 & NA\\
\hline
Foliesteektoestel & 74 & NA\\
\hline
Grindboor & 24 & NA\\
\hline
Hamerend boorsysteem & 75 & NA\\
\hline
Handpuls & 76 & NA\\
\hline
Holle avegaar & 77 & NA\\
\hline
Jenkins-mudsampler & 78 & NA\\
\hline
Kernboor & 79 & NA\\
\hline
Mostap & 80 & NA\\
\hline
Multi-sampler & 48 & NA\\
\hline
Akkermanboor & 42 & NA\\
\hline
Pulsboor & 26 & NA\\
\hline
Sonisch boorsysteem & 81 & NA\\
\hline
Spitsmuisboor & 29 & NA\\
\hline
Valbom & 14 & NA\\
\hline
Van der Horst-steektoestel & 82 & NA\\
\hline
Veenboor & 18 & NA\\
\hline
Vibro-corer & 16 & NA\\
\hline
Vrijwitboor & 52 & NA\\
\hline
Zuigerboor & 17 & NA\\
\hline
Akoestische zandtransportmeter & 43 & NA\\
\hline
Automatisch monstername apparaat & 71 & NA\\
\hline
Dompelpomp & 35 & NA\\
\hline
Doorstroomcentrifuge & 85 & NA\\
\hline
Elektrisch schepnet & 56 & NA\\
\hline
Flushing sampler & 67 & NA\\
\hline
Handnet & 58 & NA\\
\hline
Korf & 59 & NA\\
\hline
Kuil & 60 & NA\\
\hline
Modderpuls & 47 & NA\\
\hline
Mosselkorf & 61 & NA\\
\hline
Planktonnet & 63 & NA\\
\hline
Poliepgrijper & 49 & NA\\
\hline
Pollepel & 69 & NA\\
\hline
Roset sampler & 70 & NA\\
\hline
Schietfuik & 64 & NA\\
\hline
Sedimentval & 50 & NA\\
\hline
Siliconen rubber passive sampler & 84 & NA\\
\hline
Slib sampler & 51 & NA\\
\hline
Snoeischaar & 65 & NA\\
\hline
Sondeerapparaat & 28 & NA\\
\hline
Speeddisk passive sampler & 86 & NA\\
\hline
Spiraal & 23 & NA\\
\hline
Steekbuis & 34 & NA\\
\hline
Vacuümpomp & 36 & NA\\
\hline
Cilindrisch monsternemingstoestel & 32 & NA\\
\hline
Werpkorf & 66 & NA\\
\hline
Elektrisch schepnet met keernetten & 88 & NA\\
\hline
Stortkuil & 89 & NA\\
\hline
Wonderkuil & 90 & NA\\
\hline
Zegen & 91 & NA\\
\hline
Zegen met keernetten & 92 & NA\\
\hline
Elektrostramienkor & 93 & NA\\
\hline
Atoomkuil & 97 & NA\\
\hline
Ankerkuil & 98 & NA\\
\hline
Graafmachine & 99 & NA\\
\hline
Schep & 100 & NA\\
\hline
Wortelboor & 101 & NA\\
\hline
Veenguts & 102 & NA\\
\hline
Geoprobe & 103 & NA\\
\hline
Maatbeker & 104 & NA\\
\hline
Peristaltische pomp & 105 & NA\\
\hline
Telescopische stok met flessenhouder & 106 & NA\\
\hline
Stootijzer & 107 & NA\\
\hline
Macrozoöbenthos zuiger & 109 & NA\\
\hline
\end{tabular}

\hypertarget{bemonsteringsmethode}{%
\section{Bemonsteringsmethode}\label{bemonsteringsmethode}}

\begin{tabular}[t]{l}
\hline
Omschrijving\\
\hline
HH-W10A:2010 Bemonstering van zoöplankton voor EBeo\\
\hline
HH-W11A:2010 Inventarisatie van vegetatie\\
\hline
HH-W12A:2010 Bemonstering van macrofauna\\
\hline
HH-W13A:2010 Bestandsopname van vis voor de KRW\\
\hline
HH-W7A:2010 Bemonstering van fytoplankton in opp.water\\
\hline
HH-W8A:2010 Bemonstering van sieralgen in opp.water\\
\hline
HH-W9A:2010 Bemonstering van kiezelwieren in opp.water\\
\hline
Stowa 2002-07 - HVB-AM\\
\hline
Stowa 2002-07 - HVB-BOM\\
\hline
Stowa 2002-07 - HVB-KVM\\
\hline
Stowa 2002-07 - HVB-MTM\\
\hline
NEN-EN-ISO 16665:2005 en\\
\hline
NEN-EN-ISO 19458:2007 en\\
\hline
NEN-ISO 23893-1:2007 en\\
\hline
ISO 5667-10:1992 en\\
\hline
ISO 5667-11:1993 en\\
\hline
ISO 5667-12:1995 en\\
\hline
NEN-EN-ISO 5667-13:1998 en\\
\hline
NEN-EN-ISO 5667-16:1998 en\\
\hline
ISO 5667-17:2000 en\\
\hline
ISO 5667-18:2001 en\\
\hline
NEN-EN-ISO 5667-19:2004\\
\hline
ISO 5667-4:1987 en\\
\hline
NEN-ISO 5667-5:2007\\
\hline
ISO 5667-6:2005\\
\hline
ISO 5667-7:1993\\
\hline
ISO 5667-8:1993\\
\hline
ISO 5667-9:1992\\
\hline
NEN-ISO 7828:1994 en\\
\hline
NEN-EN-ISO 9391:1995 en\\
\hline
NEN-EN 13946:2003 en\\
\hline
NEN-EN 14011:2003 en\\
\hline
NEN-EN 14184:2003 en\\
\hline
NEN-EN 14757:2005 en\\
\hline
NEN-EN 15196:2006 en\\
\hline
NEN-EN 15460:2007 en\\
\hline
NEN-EN 15708:2007 en\\
\hline
NEN 5625:2007 nl\\
\hline
NEN 6600-1:2002 nl\\
\hline
NEN 6600-2:2002 nl\\
\hline
RWSV - 913.00.B001b\\
\hline
RWSV - 913.00.B002\\
\hline
RWSV - 913.00.B004b\\
\hline
RWSV - 913.00.B005\\
\hline
RWSV - 913.00.B007\\
\hline
RWSV - 913.00.B050\\
\hline
RWSV - 913.00.B051\\
\hline
RWSV - 913.00.B200\\
\hline
RWSV - 913.00.W001\\
\hline
RWSV - 913.00.W002\\
\hline
RWSV - 913.00.W003\\
\hline
RWSV - 913.00.W005\\
\hline
RWSV - 913.00.W010\\
\hline
NEN 6600-1:2009 nl\\
\hline
NEN 6600-2:2019 nl\\
\hline
NEN 5744:1991 nl\\
\hline
NEN 5744:2011 nl\\
\hline
NEN 5744:2011/A1:2013 nl\\
\hline
NEN 5745:1997 nl\\
\hline
NTA 8017:2008 nl\\
\hline
NTA 8017:2016 nl\\
\hline
SIKB Protocol 2002, 12-12-2013\\
\hline
SIKB Protocol 2002, 1-2-2018\\
\hline
SIKB Protocol 2003 v6 1-2-2018\\
\hline
RIVM Blauwalgenprotocol 2020\\
\hline
NEN 6600-1:2019 nl\\
\hline
CMA/1/A.14 september 2020\\
\hline
NEN-EN-ISO 5667-3:2012 en\\
\hline
Leidraad voor passive sampling 2012\\
\hline
NPR 8066:2010 nl\\
\hline
Eigen methode\\
\hline
onbekend\\
\hline
\end{tabular}

Conclusie: Vooral Nederlandse methodieken. Best lastig te vergelijken met internationale codes. Even parkeren.

\hypertarget{bemonsteringssoort}{%
\section{Bemonsteringssoort}\label{bemonsteringssoort}}

\begin{tabular}[t]{l|l}
\hline
  & Omschrijving\\
\hline
5 & Mengbemonstering\\
\hline
6 & Verzamelbemonstering\\
\hline
1 & Steekbemonstering\\
\hline
2 & Volumeproportionele bemonstering\\
\hline
3 & Tijdsproportionele bemonstering\\
\hline
4 & Passieve bemonstering\\
\hline
\end{tabular}

Mappen naar welke BODC tabel?

\hypertarget{meetapparaat}{%
\section{Meetapparaat}\label{meetapparaat}}

\hypertarget{kleur}{%
\section{Kleur}\label{kleur}}

\hypertarget{statistischeparameter}{%
\section{Statistischeparameter}\label{statistischeparameter}}

\hypertarget{waardebepalingsmethode}{%
\section{Waardebepalingsmethode}\label{waardebepalingsmethode}}

\hypertarget{waardebepalingstechniek}{%
\section{Waardebepalingstechniek}\label{waardebepalingstechniek}}

\hypertarget{waardebewerkingsmethode}{%
\section{Waardebewerkingsmethode}\label{waardebewerkingsmethode}}

\hypertarget{biologie-orgaan}{%
\section{Biologie: Orgaan}\label{biologie-orgaan}}

\hypertarget{biologie-biologisch-kenmerk}{%
\section{Biologie: Biologisch Kenmerk}\label{biologie-biologisch-kenmerk}}

\hypertarget{biologie-overig}{%
\chapter{Biologie overig}\label{biologie-overig}}

Voor biologische termen zijn de BODC vocabulaires over het algemeen minder goed geschikt.

Er is een zoekmachine voor biologische ontologieën kunnen gevonden worden via \url{https://www.ebi.ac.uk/ols/index}. Mogelijk kan deze in sommige gevallen uitkomst bieden. Hieronder wordt verder ingegaan op de AQUO vocabulaires die te maken hebben met biologie.

\hypertarget{biologie-orgaan-1}{%
\section{Biologie: Orgaan}\label{biologie-orgaan-1}}

De AQUO tabel ``Orgaan'' bevat standaardnamen voor de verschillende lichaamsonderdelen en organen, waarin of waaraan metingen zijn gedaan. Er is geen corresponderende tabel in de BODC bibliotheek.

\begin{table}

\caption{\label{tab:unnamed-chunk-2}Lijst met organen in AQUO domeintabel "Orgaan"}
\centering
\begin{tabular}[t]{l|l}
\hline
  & Omschrijving\\
\hline
1 & Bloed\\
\hline
12 & Dijbeen\\
\hline
23 & Dierlijk weefsel\\
\hline
29 & Darm\\
\hline
30 & Dooier\\
\hline
31 & Embryo\\
\hline
32 & Ellepijp\\
\hline
33 & Eierschaal\\
\hline
34 & Eivlies\\
\hline
2 & Filet\\
\hline
3 & Gal\\
\hline
4 & Handwortelbeen\\
\hline
5 & Hersenen\\
\hline
6 & Hart\\
\hline
7 & Kop\\
\hline
8 & Levercel\\
\hline
9 & Lever\\
\hline
10 & Maag\\
\hline
11 & Nier\\
\hline
13 & Opperarmbeen\\
\hline
14 & Oog\\
\hline
15 & Plasma\\
\hline
16 & Penis\\
\hline
17 & Scheenbeen\\
\hline
18 & Struif\\
\hline
19 & Snavel\\
\hline
20 & Schelp\\
\hline
21 & Staart\\
\hline
22 & Spierweefsel\\
\hline
24 & genetisch gemodificeerde T47D-cellijn\\
\hline
25 & Vin\\
\hline
26 & Vlees\\
\hline
27 & Vleugel\\
\hline
28 & Bruinvlees\\
\hline
\end{tabular}
\end{table}

\hypertarget{biologie-biologisch-kenmerk-1}{%
\section{Biologie: Biologisch Kenmerk}\label{biologie-biologisch-kenmerk-1}}

BiologischKenmerk

\begin{table}

\caption{\label{tab:unnamed-chunk-3}Lijst met biologische kenmerken in AQUO domeintabel "BiologischKenmerk"}
\centering
\begin{tabular}[t]{l|l}
\hline
  & Omschrijving\\
\hline
1 & dood\\
\hline
16 & Levend\\
\hline
27 & Geslacht-Man\\
\hline
38 & Geslacht-Vrouw\\
\hline
49 & Geslacht-Onbekend\\
\hline
60 & Fytoplanktonl.klasse groter dan 10 en kleiner of gelijk 20um\\
\hline
71 & Fytoplanktonl.klasse groter dan 1 en kleiner of gelijk 2 um\\
\hline
82 & Fytoplanktonl.klasse groter dan 2 en kleiner of gelijk 5 um\\
\hline
113 & Fytoplanktonl.klasse groter dan 5 en kleiner of gelijk 10 um\\
\hline
2 & Fytoplanktonlengteklasse groter dan 20 um\\
\hline
7 & Fytoplanktonlengteklasse groter dan 5 um\\
\hline
8 & Fytoplanktonlengteklasse kleiner of gelijk aan 10 um\\
\hline
9 & Fytoplanktonlengteklasse kleiner of gelijk aan 1 um\\
\hline
10 & Fytoplanktonlengteklasse kleiner of gelijk aan 5 um\\
\hline
11 & Vislengteklasse-0 (0+: vis in het eerste levensjaar)\\
\hline
12 & Vislengteklasse-1 (ouder dan 0+ en met een lengte t/m 15 cm)\\
\hline
13 & Vislengteklasse-2 (16 t/m 25 cm)\\
\hline
14 & Vislengteklasse-3 (26 t/m 40 cm)\\
\hline
15 & Vislengteklasse-4 (groter dan 40 cm)\\
\hline
17 & Vislengte snoek klasse 1 (0 t/m 15 cm)\\
\hline
18 & Vislengte snoek klasse 2 (16 t/m 35 cm)\\
\hline
19 & Vislengte snoek klasse 3 (36 t/m 44 cm)\\
\hline
20 & Vislengte snoek klasse 4 (45 t/m 54 cm)\\
\hline
21 & Vislengte snoek klasse 5 (groter dan 54 cm)\\
\hline
22 & Zoöplanktonlengteklasse 100-360 um\\
\hline
23 & Zoöplanktonlengteklasse  50-100 um\\
\hline
24 & Zoöplanktonlengteklasse groter dan 100 um\\
\hline
25 & Zoöplanktonlengteklasse groter dan 1 mm\\
\hline
26 & Zoöplanktonlengteklasse groter dan 360 um\\
\hline
28 & Zoöplanktonlengteklasse kleiner dan 1 mm\\
\hline
29 & Levensstadium-Adult\\
\hline
30 & Levensstadium-Cercarie\\
\hline
31 & Levensstadium-Copepodiet\\
\hline
32 & Levensstadium-Dauer larve\\
\hline
33 & Levensstadium-Ei\\
\hline
34 & Levensstadium-Eerste levensjaar\\
\hline
35 & Levensstadium-Embryo\\
\hline
36 & Levensstadium-Ephippium\\
\hline
37 & Levensstadium-Exuvium\\
\hline
39 & Levensstadium-Flagellaat\\
\hline
40 & Levensstadium-Juveniel\\
\hline
41 & Levensstadium-Kuiken\\
\hline
42 & Levensstadium-Larve\\
\hline
43 & Levensstadium-Nauplius\\
\hline
44 & Levensstadium-Nimf\\
\hline
45 & Levensstadium-Onvolwassen\\
\hline
46 & Levensstadium-Pop\\
\hline
47 & Levensstadium-Pul of donsjong\\
\hline
48 & Levensstadium-Spore\\
\hline
50 & Levensstadium-Veliger\\
\hline
51 & Levensstadium-Onbekend\\
\hline
52 & Energiebron-Fototroof\\
\hline
53 & Energiebron-Heterotroof\\
\hline
54 & Levensvorm-Cel (los)\\
\hline
55 & Levensvorm-Coenobium\\
\hline
56 & Levensvorm-Filament\\
\hline
57 & Levensvorm-Kolonie\\
\hline
58 & Levensvorm-Kolonie 2 tot 4 cellen\\
\hline
59 & Levensvorm-Kolonie 5 tot 10 cellen\\
\hline
61 & Levensvorm-Kolonie kleiner dan 2 cellen\\
\hline
62 & Levensvorm-Solitair\\
\hline
63 & Verschijningsvorm-Behaard\\
\hline
64 & Verschijningsvorm-Gepantserd\\
\hline
65 & Verschijningsvorm-Imposex\\
\hline
66 & Verschijningsvorm-Naakt\\
\hline
67 & Verschijningsvorm-Olie besmeurd\\
\hline
68 & Levensstadium-Cyste ruststadium bij bepaalde organismen\\
\hline
69 & Benthoslengteklasse kleiner dan 5 mm\\
\hline
70 & Benthoslengteklasse groter dan 5 en kleiner of gelijk 10 mm\\
\hline
72 & Benthoslengteklasse groter dan 10 en kleiner of gelijk 15 mm\\
\hline
73 & Benthoslengteklasse groter dan 15 en kleiner of gelijk 20 mm\\
\hline
74 & Benthoslengteklasse groter dan 20 en kleiner of gelijk 25 mm\\
\hline
75 & Benthoslengteklasse groter dan 25 en kleiner of gelijk 30 mm\\
\hline
76 & Benthoslengteklasse groter dan 30 en kleiner of gelijk 35 mm\\
\hline
77 & Benthoslengteklasse groter dan 35 en kleiner of gelijk 40 mm\\
\hline
78 & Benthoslengteklasse groter dan 40 en kleiner of gelijk 45 mm\\
\hline
79 & Benthoslengteklasse groter dan 45 en kleiner of gelijk 50 mm\\
\hline
80 & Benthoslengteklasse groter dan 50 en kleiner of gelijk 55 mm\\
\hline
81 & Benthoslengteklasse groter dan 55 en kleiner of gelijk 60 mm\\
\hline
83 & Benthoslengteklasse groter dan 60 en kleiner of gelijk 65 mm\\
\hline
104 & Benthoslengteklasse groter dan 65 en kleiner of gelijk 70 mm\\
\hline
105 & Benthoslengteklasse groter dan 70 en kleiner of gelijk 75 mm\\
\hline
106 & Benthoslengteklasse groter dan 75 en kleiner of gelijk 80 mm\\
\hline
107 & Benthoslengteklasse groter dan 80 en kleiner of gelijk 85 mm\\
\hline
108 & Benthoslengteklasse groter dan 85 en kleiner of gelijk 90 mm\\
\hline
109 & Benthoslengteklasse groter dan 90 en kleiner of gelijk 95 mm\\
\hline
110 & Benthosl.klasse groter dan 95 en kleiner of gelijk 100 mm\\
\hline
111 & Benthosl.klasse groter dan 100 en kleiner of gelijk 105 mm\\
\hline
112 & Benthosl.klasse groter dan 105 en kleiner of gelijk 110 mm\\
\hline
114 & Benthosl.klasse groter dan 110 en kleiner of gelijk 115 mm\\
\hline
214 & Benthosl.klasse groter dan 115 en kleiner of gelijk 120 mm\\
\hline
315 & Benthosl.klasse groter dan 120 en kleiner of gelijk 125 mm\\
\hline
317 & Benthosl.klasse groter dan 125 en kleiner of gelijk 130 mm\\
\hline
318 & Benthosl.klasse groter dan 130 en kleiner of gelijk 135 mm\\
\hline
319 & Benthosl.klasse groter dan 135 en kleiner of gelijk 140 mm\\
\hline
320 & Benthosl.klasse groter dan 140 en kleiner of gelijk 145 mm\\
\hline
321 & Benthosl.klasse groter dan 145 en kleiner of gelijk 150 mm\\
\hline
322 & Benthoslengteklasse groter dan 150 mm\\
\hline
323 & Benthoslengteklasse kleiner of gelijk aan 7mm\\
\hline
3 & Benthoslengteklasse groter dan 7mm\\
\hline
4 & Benthoslengteklasse groter dan 5 en kleiner of gelijk 20 mm\\
\hline
5 & Benthoslengteklasse groter dan 20 en kleiner of gelijk 30 mm\\
\hline
6 & Benthoslengteklasse groter dan 30 en kleiner of gelijk 40 mm\\
\hline
84 & Macrofaunalengteklasse 1 cm\\
\hline
85 & Macrofaunalengteklasse 2 cm\\
\hline
86 & Macrofaunalengteklasse 3 cm\\
\hline
87 & Macrofaunalengteklasse 4 cm\\
\hline
88 & Macrofaunalengteklasse 5 cm\\
\hline
89 & Macrofaunalengteklasse 6 cm\\
\hline
90 & Macrofaunalengteklasse 7 cm\\
\hline
91 & Macrofaunalengteklasse 8 cm\\
\hline
92 & Macrofaunalengteklasse 9 cm\\
\hline
93 & Macrofaunalengteklasse 10 cm\\
\hline
94 & Macrofaunalengteklasse 11 cm\\
\hline
95 & Macrofaunalengteklasse 12 cm\\
\hline
96 & Macrofaunalengteklasse 13 cm\\
\hline
97 & Macrofaunalengteklasse 14 cm\\
\hline
98 & Macrofaunalengteklasse 15 cm\\
\hline
99 & Macrofaunalengteklasse 16 cm\\
\hline
100 & Macrofaunalengteklasse 17 cm\\
\hline
101 & Macrofaunalengteklasse 18 cm\\
\hline
102 & Macrofaunalengteklasse 19 cm\\
\hline
103 & Macrofaunalengteklasse 20 cm\\
\hline
115 & Vislengteklasse 1 cm\\
\hline
116 & Vislengteklasse 2 cm\\
\hline
117 & Vislengteklasse 3 cm\\
\hline
118 & Vislengteklasse 4 cm\\
\hline
119 & Vislengteklasse 5 cm\\
\hline
120 & Vislengteklasse 6 cm\\
\hline
121 & Vislengteklasse 7 cm\\
\hline
122 & Vislengteklasse 8 cm\\
\hline
123 & Vislengteklasse 9 cm\\
\hline
124 & Vislengteklasse 10 cm\\
\hline
125 & Vislengteklasse 11 cm\\
\hline
126 & Vislengteklasse 12 cm\\
\hline
127 & Vislengteklasse 13 cm\\
\hline
128 & Vislengteklasse 14 cm\\
\hline
129 & Vislengteklasse 15 cm\\
\hline
130 & Vislengteklasse 16 cm\\
\hline
131 & Vislengteklasse 17 cm\\
\hline
132 & Vislengteklasse 18 cm\\
\hline
133 & Vislengteklasse 19 cm\\
\hline
134 & Vislengteklasse 20 cm\\
\hline
135 & Vislengteklasse 21 cm\\
\hline
136 & Vislengteklasse 22 cm\\
\hline
137 & Vislengteklasse 23 cm\\
\hline
138 & Vislengteklasse 24 cm\\
\hline
139 & Vislengteklasse 25 cm\\
\hline
140 & Vislengteklasse 26 cm\\
\hline
141 & Vislengteklasse 27 cm\\
\hline
142 & Vislengteklasse 28 cm\\
\hline
143 & Vislengteklasse 29 cm\\
\hline
144 & Vislengteklasse 30 cm\\
\hline
145 & Vislengteklasse 31 cm\\
\hline
146 & Vislengteklasse 32 cm\\
\hline
147 & Vislengteklasse 33 cm\\
\hline
148 & Vislengteklasse 34 cm\\
\hline
149 & Vislengteklasse 35 cm\\
\hline
150 & Vislengteklasse 36 cm\\
\hline
151 & Vislengteklasse 37 cm\\
\hline
152 & Vislengteklasse 38 cm\\
\hline
153 & Vislengteklasse 39 cm\\
\hline
154 & Vislengteklasse 40 cm\\
\hline
155 & Vislengteklasse 41 cm\\
\hline
156 & Vislengteklasse 42 cm\\
\hline
157 & Vislengteklasse 43 cm\\
\hline
158 & Vislengteklasse 44 cm\\
\hline
159 & Vislengteklasse 45 cm\\
\hline
160 & Vislengteklasse 46 cm\\
\hline
161 & Vislengteklasse 47 cm\\
\hline
162 & Vislengteklasse 48 cm\\
\hline
163 & Vislengteklasse 49 cm\\
\hline
164 & Vislengteklasse 50 cm\\
\hline
165 & Vislengteklasse 51 cm\\
\hline
166 & Vislengteklasse 52 cm\\
\hline
167 & Vislengteklasse 53 cm\\
\hline
168 & Vislengteklasse 54 cm\\
\hline
169 & Vislengteklasse 55 cm\\
\hline
170 & Vislengteklasse 56 cm\\
\hline
171 & Vislengteklasse 57 cm\\
\hline
172 & Vislengteklasse 58 cm\\
\hline
173 & Vislengteklasse 59 cm\\
\hline
174 & Vislengteklasse 60 cm\\
\hline
175 & Vislengteklasse 61 cm\\
\hline
176 & Vislengteklasse 62 cm\\
\hline
177 & Vislengteklasse 63 cm\\
\hline
178 & Vislengteklasse 64 cm\\
\hline
179 & Vislengteklasse 65 cm\\
\hline
180 & Vislengteklasse 66 cm\\
\hline
181 & Vislengteklasse 67 cm\\
\hline
182 & Vislengteklasse 68 cm\\
\hline
183 & Vislengteklasse 69 cm\\
\hline
184 & Vislengteklasse 70 cm\\
\hline
185 & Vislengteklasse 71 cm\\
\hline
186 & Vislengteklasse 72 cm\\
\hline
187 & Vislengteklasse 73 cm\\
\hline
188 & Vislengteklasse 74 cm\\
\hline
189 & Vislengteklasse 75 cm\\
\hline
190 & Vislengteklasse 76 cm\\
\hline
191 & Vislengteklasse 77 cm\\
\hline
192 & Vislengteklasse 78 cm\\
\hline
193 & Vislengteklasse 79 cm\\
\hline
194 & Vislengteklasse 80 cm\\
\hline
195 & Vislengteklasse 81 cm\\
\hline
196 & Vislengteklasse 82 cm\\
\hline
197 & Vislengteklasse 83 cm\\
\hline
198 & Vislengteklasse 84 cm\\
\hline
199 & Vislengteklasse 85 cm\\
\hline
200 & Vislengteklasse 86 cm\\
\hline
201 & Vislengteklasse 87 cm\\
\hline
202 & Vislengteklasse 88 cm\\
\hline
203 & Vislengteklasse 89 cm\\
\hline
204 & Vislengteklasse 90 cm\\
\hline
205 & Vislengteklasse 91 cm\\
\hline
206 & Vislengteklasse 92 cm\\
\hline
207 & Vislengteklasse 93 cm\\
\hline
208 & Vislengteklasse 94 cm\\
\hline
209 & Vislengteklasse 95 cm\\
\hline
210 & Vislengteklasse 96 cm\\
\hline
211 & Vislengteklasse 97 cm\\
\hline
212 & Vislengteklasse 98 cm\\
\hline
213 & Vislengteklasse 99 cm\\
\hline
215 & Vislengteklasse 100 cm\\
\hline
216 & Vislengteklasse 101 cm\\
\hline
217 & Vislengteklasse 102 cm\\
\hline
218 & Vislengteklasse 103 cm\\
\hline
219 & Vislengteklasse 104 cm\\
\hline
220 & Vislengteklasse 105 cm\\
\hline
221 & Vislengteklasse 106 cm\\
\hline
222 & Vislengteklasse 107 cm\\
\hline
223 & Vislengteklasse 108 cm\\
\hline
224 & Vislengteklasse 109 cm\\
\hline
225 & Vislengteklasse 110 cm\\
\hline
226 & Vislengteklasse 111 cm\\
\hline
227 & Vislengteklasse 112 cm\\
\hline
228 & Vislengteklasse 113 cm\\
\hline
229 & Vislengteklasse 114 cm\\
\hline
230 & Vislengteklasse 115 cm\\
\hline
231 & Vislengteklasse 116 cm\\
\hline
232 & Vislengteklasse 117 cm\\
\hline
233 & Vislengteklasse 118 cm\\
\hline
234 & Vislengteklasse 119 cm\\
\hline
235 & Vislengteklasse 120 cm\\
\hline
236 & Vislengteklasse 121 cm\\
\hline
237 & Vislengteklasse 122 cm\\
\hline
238 & Vislengteklasse 123 cm\\
\hline
239 & Vislengteklasse 124 cm\\
\hline
240 & Vislengteklasse 125 cm\\
\hline
241 & Vislengteklasse 126 cm\\
\hline
242 & Vislengteklasse 127 cm\\
\hline
243 & Vislengteklasse 128 cm\\
\hline
244 & Vislengteklasse 129 cm\\
\hline
245 & Vislengteklasse 130 cm\\
\hline
246 & Vislengteklasse 131 cm\\
\hline
247 & Vislengteklasse 132 cm\\
\hline
248 & Vislengteklasse 133 cm\\
\hline
249 & Vislengteklasse 134 cm\\
\hline
250 & Vislengteklasse 135 cm\\
\hline
251 & Vislengteklasse 136 cm\\
\hline
252 & Vislengteklasse 137 cm\\
\hline
253 & Vislengteklasse 138 cm\\
\hline
254 & Vislengteklasse 139 cm\\
\hline
255 & Vislengteklasse 140 cm\\
\hline
256 & Vislengteklasse 141 cm\\
\hline
257 & Vislengteklasse 142 cm\\
\hline
258 & Vislengteklasse 143 cm\\
\hline
259 & Vislengteklasse 144 cm\\
\hline
260 & Vislengteklasse 145 cm\\
\hline
261 & Vislengteklasse 146 cm\\
\hline
262 & Vislengteklasse 147 cm\\
\hline
263 & Vislengteklasse 148 cm\\
\hline
264 & Vislengteklasse 149 cm\\
\hline
265 & Vislengteklasse 150 cm\\
\hline
266 & Vislengteklasse 151 cm\\
\hline
267 & Vislengteklasse 152 cm\\
\hline
268 & Vislengteklasse 153 cm\\
\hline
269 & Vislengteklasse 154 cm\\
\hline
270 & Vislengteklasse 155 cm\\
\hline
271 & Vislengteklasse 156 cm\\
\hline
272 & Vislengteklasse 157 cm\\
\hline
273 & Vislengteklasse 158 cm\\
\hline
274 & Vislengteklasse 159 cm\\
\hline
275 & Vislengteklasse 160 cm\\
\hline
276 & Vislengteklasse 161 cm\\
\hline
277 & Vislengteklasse 162 cm\\
\hline
278 & Vislengteklasse 163 cm\\
\hline
279 & Vislengteklasse 164 cm\\
\hline
280 & Vislengteklasse 165 cm\\
\hline
281 & Vislengteklasse 166 cm\\
\hline
282 & Vislengteklasse 167 cm\\
\hline
283 & Vislengteklasse 168 cm\\
\hline
284 & Vislengteklasse 169 cm\\
\hline
285 & Vislengteklasse 170 cm\\
\hline
286 & Vislengteklasse 171 cm\\
\hline
287 & Vislengteklasse 172 cm\\
\hline
288 & Vislengteklasse 173 cm\\
\hline
289 & Vislengteklasse 174 cm\\
\hline
290 & Vislengteklasse 175 cm\\
\hline
291 & Vislengteklasse 176 cm\\
\hline
292 & Vislengteklasse 177 cm\\
\hline
293 & Vislengteklasse 178 cm\\
\hline
294 & Vislengteklasse 179 cm\\
\hline
295 & Vislengteklasse 180 cm\\
\hline
296 & Vislengteklasse 181 cm\\
\hline
297 & Vislengteklasse 182 cm\\
\hline
298 & Vislengteklasse 183 cm\\
\hline
299 & Vislengteklasse 184 cm\\
\hline
300 & Vislengteklasse 185 cm\\
\hline
301 & Vislengteklasse 186 cm\\
\hline
302 & Vislengteklasse 187 cm\\
\hline
303 & Vislengteklasse 188 cm\\
\hline
304 & Vislengteklasse 189 cm\\
\hline
305 & Vislengteklasse 190 cm\\
\hline
306 & Vislengteklasse 191 cm\\
\hline
307 & Vislengteklasse 192 cm\\
\hline
308 & Vislengteklasse 193 cm\\
\hline
309 & Vislengteklasse 194 cm\\
\hline
310 & Vislengteklasse 195 cm\\
\hline
311 & Vislengteklasse 196 cm\\
\hline
312 & Vislengteklasse 197 cm\\
\hline
313 & Vislengteklasse 198 cm\\
\hline
314 & Vislengteklasse 199 cm\\
\hline
316 & Vislengteklasse 200 cm\\
\hline
\end{tabular}
\end{table}

De BODC vocabulaire die hier het meest mee verwant is \href{https://vocab.seadatanet.org/v_bodc_vocab_v2/search.asp?lib=S11}{S11}. Deze bevat voornamelijk grootteklassen voor zoöplankton alsmede algemene namen voor biologische ontwikkelingsstadia.

Op dit moment is geen poging gedaan om een mapping te maken. Het is waarschijnlijk nodig om bijna alle AQUO termen aan te vragen als uitbreiding op de S11 tabel.

  \bibliography{book.bib,packages.bib}

\end{document}
